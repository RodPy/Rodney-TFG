% [ML] FECHA DE REVISIÓN 01/09/20
% [ML] ojo: correcciones abajo de cada párrafo
% [ML] Leer el párrafo y luego leer mis anotaciones completas de abajo. En algunas ocasiones tendrías que dar una segunda leída a mis comentarios ya que voy linea por linea en el párrafo. En otras son explicaciones generales y de una leída entenderías que quiero decir para la corrección
% [ML] Para versión 2, eliminar estos comentarios y poner los tuyos si es necesario aclarar cierta corrección
% [ML] Para lo que escribas desde donde estés ahora mismo en adelante, aunque suene exageradamente obvio, poner abajo de cada párrafo una breve y concisa explicación de que queres explicar en ese párrafo en particular, eso me permite entender bien tu punto y poder tirarte ideas de como encarar el párrafo ya que ahora estoy muy limitado a correcciones cortas y medias. A veces capaz sea necesario cambiar completamente el párrafo haciendo esa idea que tenés más clara, aunque en este capitulo no vi necesario eso solo en algunos puntos donde deje explícitos los párrafos a modo de sugerencia (te pido la idea central, porque a veces se me puede ocurrir otro párrafo distinto partiendo de ahí, y podemos jugar con esas correcciones). Usar tus iniciales entre corchetes para tus comentarios
%[ML] Las demás revisiones no serán con esta extensión, ya que tendrías que estar acostumbrándote ya a ciertas cosas, y tampoco estaré dando todas estas acotaciones sobre el método de corrección que me parece más cómodo para ambos y cosas así

\chapter[Capítulo 1. Introducción]{Introducción}
\pagestyle{fancy}

\section{Introducción}

El agua dulce es el recurso más importante para la humanidad, es un recurso indispensable de todas las actividades sociales, económicas y ambientales, es una condición para toda la vida en nuestro planeta.
%[ML] No repetir en lo posible la misma palabra en una línea (en este caso recurso), cambiar por sinónimo. ej: "El agua es el recurso más importante para la humanidad el cual es indispensable. A veces no es necesario un sustantivo nuevamente para continuar describiendo atributos del objeto.
%[ML] La palabra "condición" no es adecuada, ya que es un elemento tangible. Reemplazar por otra palabra atendiendo no caer en redundancia (recurso). Ej: económicas, ambientales, siendo un elemento para la vida (creo que no es necesaria "toda", se entiende)
Debido a la contaminación de múltiples fuentes se generan impactos que alteran la composición y calidad de los recursos hídricos causando daños al medio ambiente y a los ecosistemas en general, todo esto conlleva a una serie de consecuencias y daños de carácter reversibles e irreversibles.
% [ML] Creo que falta cierto nexo entre el párrafo anterior y el actual. pero no se exactamente como unir. podría ser algo como. Dejar nomas así y vemos en siguientes revisiones si no se te ocurre algo.

En el pa\'is, actualmente hay una gran cantidad de f\'abricas e industrias que expulsan sus aguas residuales a los afluentes sin previos trabajos de depuraci\'on en plantas de tratamientos u otros m\'etodos, por un gran d\'eficit de fiscalizaci\'on ambiental.
% [ML] Reemplazar la palabra "aguas residuales" por "desechos Residuales". Creo que da mas impacto jeje.
% [ML] Podría ser: plantas de tratamiento u otros métodos, avalado por un gran déficit... Ya que la falta de tratamiento no es una consecuencia directa del déficit, más bien "sigue existiendo" por este déficit

%
%
%PENDIENTE Hablar de sistemas de medición de de calidad del agua. PENDIENTE
% [ML] OKI
%
%

El presente trabajo final de grado (TFG) proponer el dise\~no y fabricación de una sonda, aplicado al monitoreo de calidad del agua mediante sensores que mida el potencial del hidr\'ogeno (PH), temperatura (T), oxigeno disuelto(DO), potencial de oxido reducci\'on (OPR), conductividad el\'ctrica (CE).
% [ML] que mida -> para la medición del potencial de hidrógeno (suena mas cool si no esta conjugado)
% [ML] por que quiero medir estos parámetros?, capaz se explique posteriormente en detalle, pero seria bueno una breve idea. Ej: Los cuales son parámetros de vital importancia para el estudio de la calidad del agua, etc, etc. OJO: atender que en el siguiente párrafo ya dice "calidad del agua" jeje. por eso este es un ejemplo 
 
%Hablar del proyecto PINV15-177 
En el marco del proyecto  PINV 15-177 - Veh\'iculo Aut\'onomo de Superficie (ASV) para el estudio de calidad del agua en lagos y lagunas, desarrollado en el Laboratorio de Sistemas Distribuido (LSD) perteneciente a la Facultad de Ingenier\'ia de una Universidad Nacional de Asunci\'on (FIUNA) con sede en el Centro de Innovaci\'on Tecnol\'ogica (CITEC), se pretende que de forma aut\'onoma recorra y establezca paradas para obtener lecturas de los sensores a varios niveles de profundidad de esta forma poder crear una columna de medición de los parámetros fisicoquímicos del agua georeferenciado.
% [ML] Falta un nexo con el párrafo anterior. EJ. este trabajo se enmarca en el desarrollo del proyecto PiNV5...... desarrollado en el .....
% [ML] esa parte "se pretende" quiero modificar completamente, es bueno saber que es el proyecto, pero más bueno saber que es tu aporte en el mismo. También me gustaría evitar la palabra "se pretende" y las conjugaciones (recorra y establezca). Ej: desarrollo del proyecto PINV15, desarrollado en ..... consistente en ..... el cual hace uso de la sonda para.... (igual no me gusta tanto "el uso de la sonda", por eso este es un ejemplo jeje)

El TFG  se organiza en seis cap/'itulos, desarrollados de la siguiente manera. 
El Capítulo 1, se presenta los objetivos generales y espec\'ificos del TFG,  y breve an\'alisis de la problem\'atica y se establece el alcance del trabajo realizado.
El Capítulo 2, describe los principales antecedentes sobre sobre el monitoreo de recursos h\'idricos, adem\'as de las definiciones relevantes. 
El Capítulo 3, presenta dise\~no y planificaci\'on del proyecto y sus correspondientes partes.
El Capítulo 4, presenta lo concerniente al desarrollo del sistema y la integraci\'on de todos sus componentes.   
El Capítulo 5, presenta las pruebas y resultados obtenidos de la sonda. 
Capítulo 6,  presentara las conclusiones obtenidas y recomendaciones.
% [ML] presentará-> Presenta, también se puede usar "se desarrolla", si te parece variar
% [ML] No es es que cap se habla del software OJO, o si en sistema ya se esta eso, pero esto es cosa mía, por si se te paso nomas jeje
    \section{Objetivo General}
Dise\~nar una sonda y sistema de despliegue para medici\'on a multin\'iveles de factores f\'isicoqu\'imicos de recursos h\'idricos.
%[ML] Diseñar -> Diseño e Implementación (ya que ejecutas)


\section{Objetivos Específicos}
\begin{itemize}
	\item Dise\~no mec\'anico de la sonda que contendr\'a a los sensores de CE, pH, OD, OPR, T.
    \item Investigar y fabricar  la sonda en el material apto.
    % [ML] Investigación y fabricación de la sonda. Creo que que investigación debería de estar en el item anterior, ya que uno investiga para hacer el diseño, o no se si te referis en investigación del material apto.
    \item Realizar pruebas en ambientes controlados.
    % [ML] Ejecución de pruebas
    \item Dise\~no del sistema de descenso para muestreo a multiniveles.
    \item Dise\~no de una interfaz gr\'afica y almacenamiento en base de datos.  
    \item Realizar pruebas en el lago Ypakarai.
    % [ML] Ejecución de pruebas ("de campo", a lo mejor)

\end{itemize}

\section{Problem\'atica} 

Las t\'ecnicas tradicionales de muestreo de aguas por lo general pueden resultar bastante engorrosa y lentas. En muchas ocasiones se pierde tiempo analizando lugares donde no eran necesario de forma directa, tardando en detectar a tiempo el origen de los agentes contaminantes. 
%[ML] bastante engorrosas y lentas (falto el plural). También podria ser "tradicionales de muestreo de agua resultan en un proceso bastante engorroso y lento
%[ML] "En muchas ocasiones" podría cambiarse a algo por ej, "engorrosas y lentas, perdiéndose tiempo en el análisis de zonas/áreas (queda mejor que lugares)
% [ML] "donde NO ERAN NECESARIO REALIZARSE/EJECUTARSE de forma directa" ( pienso que no queda del todo claro por la frase "no eran necesario en forma directa", el "eran" mas que nada (debía ser "era?"). 
% [ML] el tardando en detectar a tiempo el origen queda medio redundante (mucho énfasis en tres oraciones en la pérdida de tiempo), pero es importante. Usar un conector en este punto. EJEMPLO DE PÁRRAFO (PODES  USAR ESTO SI TE PARECE O VARIAR):
    % "Las técnicas tradicionales de muestro de aguas son un proceso por lo general bastante engorroso y lento, perdiéndose tiempo en el análisis de (zonas/áreas) en los cuales no era necesario aplicarlas (hablamos de las técnicas de muestreo) de forma directa. Esto a su vez resulta en el uso ineficiente del tiempo en la detección del origen de los gentes contaminantes."
% [ML] OTRA PROBLEMÁTICA? Creo que el TFG ayuda mucho en la practicidad, en el hecho de no tener que poner cada sensor de forma manual?, o también lo que seria la visualización de la información o que la misma se presenta en tiempo real (la problemática seria el tiempo que se pierde en preparar y representar los datos)

 
El presente TFG, ofrece una soluci\'on de automatizaci\'on del proceso de toma de muestras y almacenamiento en una base de datos, con una precisi\'on suficiente para poder determinar lugares donde se presentan alteraci\'on y de esta forma delimitar m\'as los muestreos con t\'ecnicas mas espec\'ificas y de esta forma hacer un muestreo mas eficiente.
%[ML] como me molesta "lugares" jaja. pero ya estoy creyendo que es cosa mía, porque tampoco suena mal 
%[ML] se presentan "alteración" (alteración de que???). 
%[ML] "delimitando los muestreos" (se delimitan los muestreos o se delimitan los lugares?)
%[ML] "alteraci\'on y de esta forma delimitar m\'as los muestreos con t\'ecnicas mas espec\'ificas y de esta forma hacer un muestreo mas eficiente". Ojo con este tipo de error. Estas repitiendo en el mismo párrafo en líneas contiguas la palabra "muestreo" y "y de esta forma". 
%[ML]  Se soluciona dando mas de una leída al párrafo completo y no por línea.
% EJEMPLO DE PÁRRAFO (PODES  USAR ESTO SI TE PARECE O VARIAR):
    %   " El presente TFG, ofrece una soluci\'on a la presente problemática mediante la automatizaci\'on del proceso de toma de muestras con precisi\'on suficiente para determinar lugares donde se presentan alteración (alteraciones?//alteración de que?) ..., delimitando (el muestreo o zona?) mediante técnicas más específicas y de esta forma hacer un estudio más eficiente. A su vez dicha información será almacenada en una base de datos para su posterior análisis y visualización"
% [ML] Sentía como que lo de base de datos estaba medio colgando en el medio.
La sonda diseñada podr\'a ser adaptado para operar con cualquier tipo de vehículo acu\'atico, y realizar\'a el muestreo en tiempo real.  

\section{Alcance}
El TFG comprende el dise\~no y fabricaci\'on de una sonda con los sensores de pH, CE,T, DO,OPR, con autonom\'ia energ\'etica propia, capaz de hacer muestreo de fluidos en tiempo real, almacenarlo en un a base de datos local y transmitir dicha información cuando este conectada a una red con internet. 
% [ML]  agregue "dicha información" después de transmitir, para ser más especifico, corregido conectado-> conectada *porque creo que hablamos de LA sonda(
El sistema de descenso incluye una gr\'ua para poder medir a multiniveles, y sensores ultras\'onicos fijos para medici/'on de profundidad y otro sensor ultras\'onico m\'ovil para detección de obst\'aculos.  
% [ML]  repetis dos veces sensores ultrasónicos. podría Ser más conciso. " incluye una grua para poder medir a multiniveles, sensores ultrasónicos fijos para la medición de profundidad y uno móvil para la detección de obstáculos" 
El modo de opraci\'on sera autom\'atico, predefinido por el usuario.


\section{Estado del arte}
% [ML]. A partir de este punto mi corrección se convierte en un blog personal. Me di cuenta que no estoy acostumbrado al concepto de "estado de arte", porque es algo que yo hacia en mis borradores digamos, y no plasmaba con esta estructura. Mi libro se baso en una estructura de capitulo 1 directo expresando mi "estado del arte" como investigación de la problemática y los conocimientos que necesitaba para usar mi tesis, terminando en los objetivos y características de diseño del aparato. Por lo que entiendo esto es como un resumen breve de la base de tu proyecto?, especifícame esto por favor apenas llegues a este punto así defino bien en mi cabeza. 
%[ML] Igual a modo de aprendizaje podes leer mis observaciones porque te hago pequeñas correcciones que te van a ser útiles, y capaz te entretengas un rato con mi confusión

%% Faltan Conectores 

Tomando como referencia los trabajos realizados por Hitz y otros \cite{hitz2012design}, en el cual presentan un novedoso buque de superficie autónomo (ASV, del ingl\'es Autonomous Surface Vehicles) dise\~nado y fabricado especialmente para el monitoreo de los recursos h\'idricos, recursos que se enfrentan a la creciente amenaza de la proliferaci\'on masiva (florac\'on) de cianobacterias nocivas. La sonda debe de ser intuitiva y de f\'acil manejo, aplicado al monitoreo de calidad del agua mediante sensores que midan el PH, la temperatura, el ox\'igeno disuelto,  ORP, conductividad el\'ectrica y color RGB. 
% [ML]  soy fanático de usar "et al " en vez de "y otros", pero creo que no está mal
% me gusta como en este párrafo se repite "recursos", pero como ves, no queda mal durante la lectura, esto pasa porque en ambas ocasiones haces referencia al mismo objeto (no se si me explico bien)
% Queda mejor referenciar en presente y no en pasado (ya hice las correcciones), podes leer como estaba en tu versión 0, así notas la diferencia.

Albarrac\'in, A. y otros \cite{samaniegodevelopment}, especifican que el desarrollo de este sistema de monitorizaci\'on de la calidad del agua permite conocer y almacenar los datos recolectados de los sitios remotos en tiempo real considerando la sincronizaci\'on entre el tiempo de sleep de los m\'odulos y el tiempo de respuesta de los sensores. 

%[ML]  lo mismo que el párrafo anterior, mejor en presente ("permitió" cambie a "permite")
%[ML]  poner "sleep" en cursiva, hacer lo mismo con palabras en idioma extranjero que uses a lo largo del libro, 
%%[ML]  No me queda del todo claro la idea del párrafo, te basas en lo que hicieron ellos?, o decís que su sistema tiene esas características y te es relevante? o es por los conectores que decis? me falta mas café?. Especifícame esto cuando llegues a este punto por whatsapp así entiendo bien y hacemos alguna corrección si es pertinente. Pero en general, el párrafo esta muy bien escrito. Ojo que esto podría influir en poner en presente o pasado los verbos 

Torres, D. \cite{torres2009diseno} manifiesta que estudiar las t\'ecnicas de medici\'on de los par\'ametros de calidad de agua y las variaciones de los mismos, permitieron que el sistema electr\'onico dise\~nado se tomar\'a como modelo para la medici\'on de los par\'ametros de calidad de agua en cualquier punto sobre el río Cauca del Valle Del Cauca (Colombia).
%[ML] el "permitieron que el sistema ..... se tomara" no me queda del todo claro, su sistema?, tu sistema?, te basas en eso?, me genera la misma sensación que el párrafo anterior. Capaz sea por el primer párrafo que tenia la estructura de "fulanito pilló esto entonces mi sonda tiene que Ser así", y estaba esperando la misma estructura.

R.G. Jones \cite{jones2002measurements}, ha desarrollado un sistema de referencia para la medici\'on de la conductividad del agua, el m\'etodo se basa en la medici\'on de la resistencia de una columna de agua de dimensiones conocidas con precisión. Hay un efecto de polarización del electrodo y la convención es extrapolar la conductividad en funci\'on de la frecuencia inversa para encontrar el valor en frecuencia inversa cero. Las mediciones pueden realizarse con una incertidumbre de aproximadamente 0,14.
%[ML] ha desarrollado --- "desarrolla..."
%[ML] "hay" podría reemplazarse por "existe"
%[ML] "frecuencia inversa " dos veces, no esta mal pero podría ser. "encontrar valor donde la misma sea cero"
%[ML] Sigo sin entender si es una especie de exposición de antecedentes, o es cosas que aprovechas en tu proyecto y necesitas que se sepa?, Eso no se debería hacer en el capitulo 2?. #Confundide

Casper y otros\cite{casper2007combining}, realizaron el seguimiento y la evaluaci\'on, especialmente la identificaci\'on de patrones o tendencias espaciales en la qu\'imica del agua (temperatura, conductividad, salinidad, turbidez, clorofila, materia org\'anica disuelta y los gases disueltos) con el uso de una innovadora combinaci\'on de veh\'iculos no tripulados de superficie (USV) y t\'ecnicas de an\'alisis geoespaciales a modo de demostrar que la percepci\'on de que los par\'ametros de muestreo y an\'alisis a intervalos regularmente espaciados sobre la superficie de un sistema fluvial ser\'an representativos de las tendencias generales, sin embargo, la estrategia de muestreo est\'andar asume tanto que los par\'ametros cambiar\'a de una manera longitudinal y aguas abajo y que la media de un par\'ametro es el nivel donde se producen impactos negativos. 
%[ML] "realizaron" --- "realizan"
%[ML] no debería ser "realizan el seguimiento, evaluación y especialmente la medición de..."
%[ML] Estoy cayendo en ficha de que esto es una breve explicación de los antecedentes y conocimientos, pero sigo pensando que esto debería estar en el cap2?. Por algo también esto está acá y te aceptaron, necesito entender el por qué nomas si después en teoría se debe citar esto, o no? #Confundide

Li, Meilin y otros\cite{li2012design}, presentan en este trabajo, el dise\~no y la implementaci\'on de un nuevo sistema USV reconstruido desde una lancha para proporcionar comodidad para los experimentos de control de aut\'onomas de futuros. El nuevo sistema USV dise\~nado se compone de tres partes: cuerpo principal, el sistema de control de ordenador de a bordo y el sistema de control de tierra. 
%[ML] No presenten en "este" trabajo. Directamente "presentan el diseño..."
%[ML] Ahora estoy pensando que es como una evolución de los sistemas que se fueron desarrollando o algo por el estilo

 Mastmija y otros\cite{masmitja2010development}, proponen en este trabajo, el desarrollo de un sistema de control para un veh\'iculo submarino autónomo dedicado a la observaci\'on de los oc\'eanos. El veh\'iculo, un h\'ibrido entre veh\'iculos submarinos aut\'onomos (AUV) y Veh\'iculos de superficie Aut\'onomo (ASV), se mueve sobre la superficie del mar y hace inmersiones verticales para obtener perfiles de una columna de agua, de acuerdo con un plan preestablecido. El desplazamiento del veh\'iculo en la superficie permite la navegaci\'on por GPS y la comunicaci\'on de telemetr\'ia por radio-modem. El sistema de control est\'a basado en un equipo integrado est\'a dise\~nado y desarrollado para este veh\'iculo que permite la navegaci\'n aut\'onoma de un veh\'iculo. Este sistema de control se ha dividido en subsistemas de navegaci\'on, propulsi\'on, de seguridad y de adquisici\'on de datos. Dado su alto rendimiento, la incorporaci\'on de algoritmos de control de trayectoria es factible. Tambi\'en, hardware y software espec\'ificos dise\~nados para el correcto funcionamiento de los sensores y propulsores.
 %[ML] eliminar "en este trabajo", o no se si te referis a su trabajo cuando decis "en este trabajo". Japiro, proponen directo.

Andrew J. y otros \cite{skinner2006using}, han desarrollado un sensor de temperatura de bajo costo que puede ser configurado en "cadenas" sumergibles distribuyé\'endolos a lo largo de un simple cable de tres hilos. El sensor ha demostrado ser capaz de entregar las mediciones de temperatura altamente coincidentes simult\'aneas a una resoluci\'on de unas pocas mil\'esimas de un grado, con una coincidencia mejor que 0.01  \textsuperscript{o}C y una incertidumbre de la medici\'on de aproximadamente 0,05  \textsuperscript{o}C. Una t\'ecnica para generar un recuento estandarizado frente a la curva de temperatura se ha desarrollado utilizando el m\'etodo de diferencias finitas.
%[ML] desarrollan, aunque a esta altura ya estoy dudando si es mejor poner en presente o pasado jajaj

Borden  y otros \cite{borden2012long},  describieron en este documento un m\'etodo para emplear m\'odems ac\'usticos subacu\'aticos para el intercambio de datos e informaci\'on  de control entre un AUV sumergida y el operador. Ejemplos de datos que ser\'ian de  inter\'es para un operador AUV podr\'ian consistir en: estado de carga de la bater\'ia, el rumbo del veh\'iculo, la profundidad del veh\'iculo, y la distancia acumulada recorrida. Este  tipo de datos puede ser transferido a través de comunicaciones acústicas subacu\'aticas, eliminando la  necesidad de que el veh\'iculo a la superficie de forma rutinaria y transmitir estos datos a trav\'es del aire. M\'odems ac\'usticos subacuáticos se pueden emplear de baja frecuencia (9 a 13 kHz) ac\'ustica para lograr  una comunicación efectiva. Las distancias de transmisi\'on de m\'as de 5000 metros se  pueden lograr con frecuencias en este rango.
% [ML] ya ni te digo lo del pasado

 P.Ramos y Otros \cite{ramos2008four} se presenta un nuevo sensor de cuatro electrodos para mediciones de conductividad del agua. Adem\'as del sensor en s\'i, todo el acondicionamiento de la se\~nal se implementa junto con el procesamiento de la se\'~nal de las salidas del sensor para determinar la conductividad del agua. El sensor est\'a dise\~nado para mediciones de conductividad en el rango de 50 mS / m hasta 5 S / m a trav\'es de la colocaci\'on correcta de los cuatro electrodos dentro del tubo por donde fluye el agua. El prototipo implementado es capaz de suministrar al sensor la corriente necesaria a la frecuencia de medici\'on, adquiriendo las señales sinusoidales a trav\'es de los electrodos de voltaje del sensor y a trav\'es de una impedancia de muestreo para determinar la corriente. Tambi\'en se incluye un sensor de temperatura en el sistema para medir la temperatura del agua y, por lo tanto, compensar la dependencia de la temperatura de la conductividad del agua.
 %[ML] "se presenta"-- "presentan", o en todo caso sería "en el trabajo de fulanito y amigos, se presenta"
 %[ML] independientemente de que lo que sea esta sección en un parrafo usas un tiempo verbal y en el siguiente usas otro, atender ese detalle
 