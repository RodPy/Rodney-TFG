% [RR] FECHA DE HOY jaja 28/09/ 20 :(


\chapter[Introducción]{Introducción}
\pagestyle{fancy}

%\section{Introducción}

El agua es un elemento \'unico de la naturaleza \cite{caribbean_seguridad_2020}, el agua dulce es el recurso natural importante,indispensable de todas las actividades sociales, económicas y ambientales, es una condición para toda la vida en nuestro planeta.
Se lo puede encontrar en la naturaleza en sus tres estados (sólido, liquido y gaseoso) y conforma la  mayor parte de la superficie del planeta. 
Desde el inicio de los tiempos estuvimos en contacto con el agua, como fuente de hidratación, medio de transporte, agricultura, ganadería, materia prima, etc., con el correr de los tiempos pudimos observar como el agua ha sufrido una serie de alteraciones debido a la explotación abusiva por estas actividades. 
Hace relativamente poco tiempo le hemos dado la importancia para estudiar la forma como impactamos y como altera el ecosistema natural, regular varios aspectos con tratados nacionales, internacionales, con el fin de preservar y no alterar mucho los ecosistemas naturales. 
Desde el a\~no 2015 los líderes mundiales adoptaron 17 Objetivos de Desarrollo Sostenible (ODS) para proteger el planeta, luchar contra la pobreza y tratar de erradicarla con el objetivo de construir un mundo más próspero,  justo y sostenible para las generaciones futuras \cite{bbva_que_nodate}. Entre los objetivos de desarrollo, se encuentra el agua, especificamente en el ODS 6, denominado Agua Limpia y saneamiento, cuyo objetivo es Garantizar la disponibilidad de agua y su gestión sostenible y el saneamiento para todos, a trav\'es de de siete objetivos \cite{onu_agua_nodate} 
6.1  De aquí a 2030, lograr el acceso universal y equitativo al agua potable a un precio asequible para todos.

6.2  De aquí a 2030, lograr el acceso a servicios de saneamiento e higiene adecuados y equitativos para todos y poner fin a la defecación al aire libre, prestando especial atención a las necesidades de las mujeres y las niñas y las personas en situaciones de vulnerabilidad

6.3  De aquí a 2030, mejorar la calidad del agua reduciendo la contaminación, eliminando el vertimiento y minimizando la emisión de productos químicos y materiales peligrosos, reduciendo a la mitad el porcentaje de aguas residuales sin tratar y aumentando considerablemente el reciclado y la reutilización sin riesgos a nivel mundial

6.4  De aquí a 2030, aumentar considerablemente el uso eficiente de los recursos hídricos en todos los sectores y asegurar la sostenibilidad de la extracción y el abastecimiento de agua dulce para hacer frente a la escasez de agua y reducir considerablemente el número de personas que sufren falta de agua

6.5  De aquí a 2030, implementar la gestión integrada de los recursos hídricos a todos los niveles, incluso mediante la cooperación transfronteriza, según proceda

6.6  De aquí a 2020, proteger y restablecer los ecosistemas relacionados con el agua, incluidos los bosques, las montañas, los humedales, los ríos, los acuíferos y los lagos

6.a  De aquí a 2030, ampliar la cooperación internacional y el apoyo prestado a los países en desarrollo para la creación de capacidad en actividades y programas relativos al agua y el saneamiento, como los de captación de agua, desalinización, uso eficiente de los recursos hídricos, tratamiento de aguas residuales, reciclado y tecnologías de reutilización

6.b  Apoyar y fortalecer la participación de las comunidades locales en la mejora de la gestión del agua y el saneamiento

%[RR]Esta primera parte lo probable es que cambie, con cosas del cap2, le agregue temporalmente el conector para mejorar su nexo 

La humanidad ha sido testigo de como se  perdieron  ecosistemas completos, extinguiéndose fauna, flora, aparici\'on, enfermedades que diezmaron poblaciones, a causa de la contaminaci\'on generada, que alteraban los ecosistemas, y esto modificaba la calidad de los recursos h\'idrico, causando da\~nos al medio ambiente y una serie de consecuencias muchas veces irreversibles. 

El Paraguay es un pa\'is rico en recursos naturales, está ubicado en el número 30, con unos 336 km$^3$ de agua dulce dentro de su territorio. 
El acuífero Guaraní es la tercera mayor reserva de agua dulce del mundo, compartido con Brasil, Argentina y Uruguay, con 70.000 km$^{2}$ de superficie en Paraguay~\cite{stp_Agua}. 
Por ello es de vital importancia que se tengan políticas publicas y departamentos que velen por esos recursos.
Es esencial el monitorio de los recursos hídricos para así detectar cambios y tomar las acciones requeridas a tiempo.

En el país, actualmente existen una gran cantidad de fábricas e industrias que expulsan sus desechos residuales a los afluentes sin previos trabajos de depuración en plantas de tratamientos u otros métodos, que no son controlados y regulados por un gran déficit de fiscalización ambiental , por la limitaciones en la mayor parte tecnol\'ogicos para poder realizar un muestreo de una forma r\'apida y confiable por los departamentos fiscalizadores competentes. 

%
%PENDIENTE Hablar de sistemas de medición de de calidad del agua. PENDIENTE
% [ML] OKI
% [RR] Creo que esto voy a hablar mejor en el cap 2
%

El presente trabajo final de grado (TFG) proponer el diseño y fabricación de una sonda, aplicado al monitoreo de calidad del agua mediante sensores, para la medición automática del potencial del hidrógeno (PH), temperatura (T), oxígeno disuelto(DO), potencial de oxido reducción (OPR), conductividad eléctrica (CE). 
Los cuales según estándares nacionales e internaciones son indicadores básicos para analizar el estado del recurso hídrico y poder inferir sobre la calidad del mismo.  


%Hablar del proyecto PINV15-177 
El presente TFG se enmarca en el proyecto PINV 15-177 - Vehículo Autónomo de Superficie (ASV) para el estudio de calidad del agua en lagos y lagunas, desarrollado en el Laboratorio de Sistemas Distribuido (LSD) perteneciente a la Facultad de Ingeniería de una Universidad Nacional de Asunción (FIUNA) con sede en el Centro de Innovación Tecnológica (CITEC). 
El ASV proporciona una automatización para el muestreo, transportando la sonda hasta la locación deseada, georeferenciar los datos obtenidos y capacidad de sensado a diferentes niveles de profundidad. 
Como mencionan \cite{guiaHANNA}en su gu\'ia, sobre la importancia de reconocer que la calidad del agua deficiente puede afectar de manera negativa  ya sea por razones naturales y/o por factores humanos, donde un monitorio regular las fuentes de agua puede ayudar a identificar problemas potenciales antes de que generen algún daño.


El TFG  se organiza en cinco capítulos, desarrollados de la siguiente manera. 
El Capítulo uno, se presentará los objetivos generales y específicos del TFG,  y breve an\'alisis de la problemática y se establece el alcance del trabajo realizado.
El Capítulo dos, describe los principales antecedentes sobre sobre el monitoreo de recursos hídricos, además de las definiciones relevantes.
El Capítulo tres, se desarrolla el diseño de las parte, lo concerniente al desarrollo del sistema y la integración de todos sus componentes.   
El Capítulo cuatro, se desarrolla las pruebas realizadas y resultados obtenidos con la sonda. 
Capítulo cinco,  presentará las conclusiones obtenidas y recomendaciones.
%[RR] Yo metí todo dentro de "sistema " realmente , podria agregar una descripción mas detallada para que se entienda lo que abarca  

\section{Objetivo General}
Diseñar e implementar una sonda y sistema de despliegue para muestreo a multiniveles de factores fisicoquímicos de recursos hídricos.


\section{Objetivos Específicos} \label{Objetivos}
\begin{itemize}
	\item Diseñar la pieza mecánica de la sonda que contendrá a los sensores de CE, pH, OD, OPR, T.
    \item Investigar y fabricar  la sonda en el material apto.
    % [RR] Yo primero hice el diseño para impresora 3D y después vi como fabricar, onda lo que queria es el motivo de la impresión 3D y comprar en el mecanizado convencional, yo use ambos 
    \item Realizar pruebas de muestreo con los sensores de la sonda en ambientes controlados.
    \item Diseñar el sistema de descenso para muestreo a multiniveles.
    \item Diseñar la interfaz gráfica y almacenamiento en base de datos.  
    \item Realizar mediciones fisicoquímicas in situ en los puntos de muestreo del lago Ypakarai.

\end{itemize}

\section{Problemática} 

 Las técnicas tradicionales de muestreo  de aguas  pueden de forma manual, resultar bastante engorrosos,lentos y costoso por los preparativos logísticos anterior y posterior a la campaña de muestro, por lo general se limitan a unas pocas muestras para su posterior análisis. 
 Estos procedimientos muchas veces enlentecen localizar rápidamente de donde provienen los contaminante y conocer le estado real del cuerpo de agua analizado. El acceso a estos datos analizados no son de fácil acceso y están acompañados de burocracia.  


% [RR] Cambie varias cosas,pendiente de mejorar mas explicando los problemas 
 
El presente TFG, pretende entregar al campo de monitoreo y sensado de recursos hídricos, un dispositivo practico, escalable, portátil, inalámbrico, de respuesta rápida, de fácil uso, que almacene los datos recolectado en una base de datos local y pueda transmitir en tiempo real a alguna alguna estación remota, generando automáticamente un reporte una vez concluida el trabajo. 

La sonda diseñada podra ser adaptado para operar con el ASV del proyecto PINV15-177, obteniendo de este sus datos georeferenciales.
Ofrecer\'a una solución de automatizaci\'on del proceso de toma de muestras, optimizando  las campañas de verificaci\'on, con menos log\'istica, m\'as r\'apidas,  mucho m\'as puntos de muestreo y generaci\'on autom\'atica de reporte cuando sea necesario.

%[RR] Reescribí todo 

\section{Estado del arte}
Tomando como referencia los trabajos realizados por Hitz y otros \cite{hitz2012design}, presentaron un novedoso buque de superficie autónomo (ASV, del ingl\'es Autonomous Surface Vehicles) que fue dise\~nado y fabricado especialmente para el monitoreo de los recursos h\'idricos, recursos que se enfrentan a la creciente amenaza de la proliferaci\'on masiva (florac\'on) de cianobacterias nocivas. La sonda debe de ser intuitiva y de f\'acil manejo, aplicado al monitoreo de calidad del agua mediante sensores que midan el PH, la temperatura, el ox\'igeno disuelto,  ORP, conductividad el\'ectrica y color RGB. 

Albarrac\'in, A. y otros \cite{samaniegodevelopment}, especifican que el desarrollo de este sistema de monitorizaci\'on de la calidad del agua permiti\'o conocer y almacenar los datos recolectados de los sitios remotos en tiempo real considerando la sincronizaci\'on entre el tiempo de sleep de los m\'odulos y el tiempo de respuesta de los sensores. 

Torres, D. \cite{torres2009diseno} manifiesta que estudiar las t\'ecnicas de medici\'on de los par\'ametros de calidad de agua y las variaciones de los mismos, permitieron que el sistema electr\'onico dise\~nado se tomar\'a como modelo para la medici\'on de los par\'ametros de calidad de agua en cualquier punto sobre el río Cauca del Valle Del Cauca (Colombia).

R.G. Jones \cite{jones2002measurements}, ha desarrollado un sistema de referencia para la medici\'on de la conductividad del agua, el m\'etodo se basa en la medici\'on de la resistencia de una columna de agua de dimensiones conocidas con precisión. Hay un efecto de polarización del electrodo y la convención es extrapolar la conductividad en funci\'on de la frecuencia inversa para encontrar el valor en frecuencia inversa cero. Las mediciones pueden realizarse con una incertidumbre de aproximadamente 0,14.

Casper y otros\cite{casper2007combining}, realizaron el seguimiento y la evaluaci\'on, especialmente la identificaci\'on de patrones o tendencias espaciales en la qu\'imica del agua (temperatura, conductividad, salinidad, turbidez, clorofila, materia org\'anica disuelta y los gases disueltos) con el uso de una innovadora combinaci\'on de veh\'iculos no tripulados de superficie (USV) y t\'ecnicas de an\'alisis geoespaciales a modo de demostrar que la percepci\'on de que los par\'ametros de muestreo y an\'alisis a intervalos regularmente espaciados sobre la superficie de un sistema fluvial ser\'an representativos de las tendencias generales, sin embargo, la estrategia de muestreo est\'andar asume tanto que los par\'ametros cambiar\'a de una manera longitudinal y aguas abajo y que la media de un par\'ametro es el nivel donde se producen impactos negativos. 

Li, Meilin y otros\cite{li2012design}, presentan en este trabajo, el dise\~no y la implementaci\'on de un nuevo sistema USV reconstruido desde una lancha para proporcionar comodidad para los experimentos de control de aut\'onomas de futuros. El nuevo sistema USV dise\~nado se compone de tres partes: cuerpo principal, el sistema de control de ordenador de a bordo y el sistema de control de tierra. 

 Mastmija y otros\cite{masmitja2010development}, proponen en este trabajo, el desarrollo de un sistema de control para un veh\'iculo submarino autónomo dedicado a la observaci\'on de los oc\'eanos. El veh\'iculo, un h\'ibrido entre veh\'iculos submarinos aut\'onomos (AUV) y Veh\'iculos de superficie Aut\'onomo (ASV), se mueve sobre la superficie del mar y hace inmersiones verticales para obtener perfiles de una columna de agua, de acuerdo con un plan preestablecido. El desplazamiento del veh\'iculo en la superficie permite la navegaci\'on por GPS y la comunicaci\'on de telemetr\'ia por radio-modem. El sistema de control est\'a basado en un equipo integrado est\'a dise\~nado y desarrollado para este veh\'iculo que permite la navegaci\'n aut\'onoma de un veh\'iculo. Este sistema de control se ha dividido en subsistemas de navegaci\'on, propulsi\'on, de seguridad y de adquisici\'on de datos. Dado su alto rendimiento, la incorporaci\'on de algoritmos de control de trayectoria es factible. Tambi\'en, hardware y software espec\'ificos dise\~nados para el correcto funcionamiento de los sensores y propulsores.

Andrew J. y otros \cite{skinner2006using}, han desarrollado un sensor de temperatura de bajo costo que puede ser configurado en "cadenas" sumergibles distribuyé\'endolos a lo largo de un simple cable de tres hilos. El sensor ha demostrado ser capaz de entregar las mediciones de temperatura altamente coincidentes simult\'aneas a una resoluci\'on de unas pocas mil\'esimas de un grado, con una coincidencia mejor que 0.01  \textsuperscript{o}C y una incertidumbre de la medici\'on de aproximadamente 0,05  \textsuperscript{o}C. Una t\'ecnica para generar un recuento estandarizado frente a la curva de temperatura se ha desarrollado utilizando el m\'etodo de diferencias finitas.

Borden  y otros \cite{borden2012long},  describieron en este documento un m\'etodo para emplear m\'odems ac\'usticos subacu\'aticos para el intercambio de datos e informaci\'on  de control entre un AUV sumergida y el operador. Ejemplos de datos que ser\'ian de  inter\'es para un operador AUV podr\'ian consistir en: estado de carga de la bater\'ia, el rumbo del veh\'iculo, la profundidad del veh\'iculo, y la distancia acumulada recorrida. Este  tipo de datos puede ser transferido a través de comunicaciones acústicas subacu\'aticas, eliminando la  necesidad de que el veh\'iculo a la superficie de forma rutinaria y transmitir estos datos a trav\'es del aire. M\'odems ac\'usticos subacuáticos se pueden emplear de baja frecuencia (9 a 13 kHz) ac\'ustica para lograr  una comunicación efectiva. Las distancias de transmisi\'on de m\'as de 5000 metros se  pueden lograr con frecuencias en este rango.

 P.Ramos y Otros \cite{ramos2008four} se presenta un nuevo sensor de cuatro electrodos para mediciones de conductividad del agua. Adem\'as del sensor en s\'i, todo el acondicionamiento de la se\~nal se implementa junto con el procesamiento de la se\'~nal de las salidas del sensor para determinar la conductividad del agua. El sensor est\'a dise\~nado para mediciones de conductividad en el rango de 50 mS / m hasta 5 S / m a trav\'es de la colocaci\'on correcta de los cuatro electrodos dentro del tubo por donde fluye el agua. El prototipo implementado es capaz de suministrar al sensor la corriente necesaria a la frecuencia de medici\'on, adquiriendo las señales sinusoidales a trav\'es de los electrodos de voltaje del sensor y a trav\'es de una impedancia de muestreo para determinar la corriente. Tambi\'en se incluye un sensor de temperatura en el sistema para medir la temperatura del agua y, por lo tanto, compensar la dependencia de la temperatura de la conductividad del agua.

\section{Alcance}
Partiendo de los trabajos de los autores anteriormente mencionados, se establecieron los alcances correspondientes al proyecto, teniendo en cuenta el lugar de implementaci\'on y capacidad logistica y financiera de contar con los equipos o herramientas necesarias .
El TFG comprende el diseño mecánico, eléctrico, electrónico, control y fabricación de una sonda, que tenga incorporados los sensores de pH, CE,T, DO, OPR, con autonomía energética propia, capaz de hacer muestreo de fluidos en tiempo real, configurados de forma manual o automático sincronizado con una ruta de algún ASV, almacenando los datos obtenidos en una base de datos local con la posibilidad de transmitirlos a una estación de base remota. 

Para el sensado a multiniveles de profundidades, se diseñó e implementó un sistema compuesto por un sonarfijo para detectar la profundidad en el punto de medición, un molinete eléctrico con la capacidad suficiente para soportar el peso de la sonda.

El sistema de descenso se comunican de forma inalámbrica con la sonda, y se pueden controlar de forma remota, siempre cuando este conectada a una red wifi con internet.

El modo de operación podría ser automático o manual, predefinido por el usuario.

%[RR]Cambie varias cosas 



%[RR] Me gustaría cerrar por de pronto este capitulo, es justo eso como mencionabas son las referencias para cuando empecé a trabajar, no tienen contexto entre sí, y supongo de ni deberían jaja, yo hice copy paste para tener más a mano cuando quiera usarlos, voy a hacer las correcciones que me sugerís . Arigato  .
