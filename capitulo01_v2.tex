% [RR] FECHA DE HOY jaja 28/09/ 20 :(


\chapter[Introducción]{Introducción}
\pagestyle{fancy}

%\section{Introducción}

El agua es un elemento \'unico de la naturaleza, \cite{caribbean_seguridad_2020} el agua dulce es el recurso natural importante, indispensable en todas las actividades sociales, económicas y ambientales, es una condición para toda la vida en nuestro planeta.
Se lo puede encontrar en la naturaleza en sus tres estados f\'isicos(sólido, l\'iquido y gaseoso), y constituye la  mayor parte de la superficie del planeta.
Hist\'oricamente estuvimos en contacto con el agua como fuente de hidratación, medio de transporte, agricultura, ganadería, materia prima, entre otros, a medida que transcurre el tiempo se puedo observar como este vital líquido, sufrió importantes alteraciones debido a la explotación irresponsables por las actividades humanas. 
Hace relativamente poco tiempo la humanidad ha comprendido la importancia de la conservación, analizando y estudiando como nuestras actividades impactan, pudiendo alterar un ecosistema.
Mediante pol\'iticas p\'ublicas, se busca regular varios aspectos con tratados nacionales e internacionales, con el fin de preservar los ecosistemas naturales.
Desde el a\~no 2015, los líderes mundiales adoptaron 17 Objetivos de Desarrollo Sostenible (ODS) para proteger el planeta, luchar contra la pobreza y tratar de erradicarla con el objetivo de construir un mundo más próspero,  justo y sostenible para las generaciones futuras, \cite{bbva_que_nodate}. 
Entre los objetivos de desarrollo se encuentra el agua, espec\'ificamente en el ODS 6, denominado Agua Limpia y saneamiento, cuyo objetivo es Garantizar la disponibilidad de agua y su gestión sostenible y el saneamiento para todos, a trav\'es de siete objetivos, \cite{onu_agua_nodate} 
, ODS 6.1  De aquí a 2030, lograr el acceso universal y equitativo al agua potable a un precio asequible para todos.
, ODS 6.2  De aquí a 2030, lograr el acceso a servicios de saneamiento e higiene adecuados y equitativos para todos y poner fin a la defecación al aire libre, prestando especial atención a las necesidades de las mujeres y las niñas y las personas en situaciones de vulnerabilidad
, ODS 6.3  De aquí a 2030, mejorar la calidad del agua, reduciendo la contaminación, eliminando el vertimiento y minimizando la emisión de productos químicos y materiales peligrosos, reduciendo a la mitad el porcentaje de aguas residuales sin tratar y aumentando considerablemente el reciclado y la reutilización sin riesgos a nivel mundial
, ODS 6.4  De aquí a 2030, aumentar considerablemente el uso eficiente de los recursos hídricos en todos los sectores y asegurar la sostenibilidad de la extracción y el abastecimiento de agua dulce para hacer frente a la escasez de agua y reducir considerablemente el número de personas que sufren falta de agua
, ODS 6.5  De aquí a 2030, implementar la gestión integrada de los recursos hídricos a todos los niveles, incluso mediante la cooperación transfronteriza, según proceda
, ODS 6.6  De aquí a 2020, proteger y restablecer los ecosistemas relacionados con el agua, incluidos los bosques, las montañas, los humedales, los ríos, los acuíferos y los lagos
, ODS 6.a  De aquí a 2030, ampliar la cooperación internacional y el apoyo prestado a los países en desarrollo para la creación de capacidad en actividades y programas relativos al agua y el saneamiento, como los de captación de agua, desalinización, uso eficiente de los recursos hídricos, tratamiento de aguas residuales, reciclado y tecnologías de reutilización
, ODS 6.b  Apoyar y fortalecer la participación de las comunidades locales en la mejora de la gestión del agua y el saneamiento,\cite{moran_agua_nodate}.
La humanidad ha sido testigo de c\'omo se ha perdido ecosistemas completos, extinguiéndose fauna y flora, por la alteraci\'on de los ecosistemas a causa de la contaminaci\'on, que modifica la calidad de los suelos y recursos h\'idrico, causando da\~nos al medio ambiente y una serie de consecuencias muchas veces irreversibles.
El Paraguay es un pa\'is rico en recursos naturales, posee unos 336 km$^3$ de agua dulce dentro de su territorio. 
El acuífero Guaraní es la tercera mayor reserva de agua dulce del mundo, compartido con Brasil, Argentina y Uruguay, con 70.000 km$^{2}$ de superficie en Paraguay,\cite{stp_Agua}. 
Por ello es de vital importancia tener e implementar políticas p\'ublicas y departamentos que protejan, regulen y preserven estos recursos.
El monitorio es esencial de los recursos hídricos para así detectar cambios y tomar las acciones requeridas a tiempo.
En el país, existen una gran cantidad de fábricas e industrias que no poseen sistemas de tratamiento de efluentes de producci\'on, vertiendo sus desechos residuales a los afluentes sin previo trabajo de depuración en planta, que no son controlados y regulados por un gran déficit de fiscalización ambiental, por las limitaciones en la mayor parte tecnol\'ogicos para poder realizar un muestreo de una forma r\'apida y confiable por los departamentos fiscalizadores competentes. 
El presente trabajo final de grado (TFG) propone el diseño y fabricación de una sonda, aplicado al monitoreo de calidad del agua mediante sensores, para la medición automática del potencial del hidrógeno (PH), temperatura (T), oxígeno disuelto(DO), potencial de \'oxido reducción (OPR) y conductividad eléctrica (CE). 
Par\'ametros que seg\'un estándares nacionales e internaciones, son indicadores básicos para analizar el estado del recurso hídrico y poder inferir sobre la calidad del mismo.  
%proyecto PINV15-177 
El TFG se enmarca en el proyecto PINV 15-177 - Vehículo Autónomo de Superficie (ASV) para el estudio de calidad del agua en lagos y lagunas, desarrollado en el Laboratorio de Sistemas Distribuido (LSD) perteneciente a la Facultad de Ingeniería de una Universidad Nacional de Asunción (FIUNA) con sede en el Centro de Innovación Tecnológica (CITEC). 
El ASV no forma parte del desarrollo de este TFG, proporciona una automatización para el muestreo, transportando la sonda hasta la locación deseada, georeferenciando los datos obtenidos, mediante una gr\'ua  capacidad de sensado a diferentes niveles de profundidad. 
Como menciona en su gu\'ia Hanna Instruments, sobre la importancia de reconocer que la calidad del agua deficiente puede afectar de manera negativa,  ya sea por razones naturales y/o por factores humanos, donde un monitoreo regular de los recursos hídricos pudiendo ayudar a identificar problemas antes de que generen algún deterioro de algún ecosistema, \cite{guiaHANNA}.
El TFG  se organiza en cinco capítulos, desarrollados de la siguiente manera: 
en el capítulo uno, se presentan los objetivos generales y específicos,  un breve an\'alisis de la problemática, y el alcance del trabajo realizado;
en el capítulo dos, se describen los antecedentes sobre el monitoreo de recursos hídricos y las definiciones relevantes;
en el capítulo tres, se desarrolla el diseño de las partes lo concerniente al desarrollo del sistema y la integración de todos sus componentes;   
en el cap\'itulo cuatro, se desarrolla las pruebas realizadas y resultados obtenidos con la sonda;
por último en el capítulo cinco,  se presentan las conclusiones obtenidas, recomendaciones, trabajos futuros y divulgaciones.
%[RR] Yo metí todo dentro de "sistema " realmente , podria agregar una descripción mas detallada para que se entienda lo que abarca  

\section{Objetivo General}
Desarrollar una sonda y sistema de despliegue para muestreo a multinivel de factores fisicoquímicos de recursos hídricos.

\section{Objetivos Específicos} \label{Objetivos}
\begin{itemize}
	\item Dise\~nar la sonda que contendrá a los sensores de CE, pH, OD, OPR, T.
    \item Investigar la sonda en el material apto.
    % [RR] Yo primero hice el diseño para impresora 3D y después vi como fabricar, onda lo que queria es el motivo de la impresión 3D y comprar en el mecanizado convencional, yo use ambos 
    \item Realizar pruebas de muestreo con los sensores de la sonda en ambientes controlados.
    \item Dise\~nar el sistema de descenso para muestreo a multinivel.
    \item Dise\~nar la interfaz gráfica y almacenamiento en base de datos.  
    \item Realizar mediciones de par\'ametros fisicoqu\'imicas in situ en los puntos de muestreo del lago Ypakarai.
\end{itemize}

\section{Problemática} 
Las técnicas tradicionales de muestreo  de aguas pueden resultar bastante engorrosos, lentos y costosos por los preparativos logísticos anteriores y posteriores a la campaña de muestreo, por lo general, se limitan a unas pocas muestras para su posterior análisis. 
Estos procedimientos muchas veces enlentecen, la localizaci\'on rápida de las fuentes de contaminación y muy pocas veces es posible conocer el estado en tiempo real del cuerpo de agua analizado. 
El TFG, pretende brindar al campo de monitoreo y sensado de recursos h\'idricos, una soluci\'on de automatizaci\'on del proceso de toma de muestras, optimizando  las campañas de verificaci\'on, con menos log\'istica, m\'as r\'apidas,  mucho m\'as puntos de muestreo y generaci\'on autom\'atica de reporte cuando sea necesario. 
Un dispositivo pr\'actico, escalable, portátil, inalámbrico, de respuesta rápida, de fácil uso, que almacene los datos recolectados en una base de datos local y pueda transmitir en tiempo real a alguna estación remota, pudiendo generar automáticamente un reporte preliminar una vez concluida el trabajo. 
La sonda diseñada podr\'a ser adaptada para operar con el ASV del proyecto PINV15-177

\section{Estado del arte}
Tomando como referencia los trabajos realizados por Hitz y otros, presentaron un novedoso buque de superficie autónomo (ASV, del ingl\'es Autonomous Surface Vehicles) que fue dise\~nado y fabricado especialmente para el monitoreo de los recursos h\'idricos, recursos que se enfrentan a la creciente amenaza de la proliferaci\'on masiva (florac\'on) de cianobacterias nocivas. La sonda debe de ser intuitiva y de f\'acil manejo, aplicado al monitoreo de calidad del agua mediante sensores que midan el PH, la temperatura, el ox\'igeno disuelto,  ORP, conductividad el\'ectrica y color RGB, \cite{hitz2012design}. 
Albarrac\'in, A. y otros, especifican que el desarrollo de este sistema de monitorizaci\'on de la calidad del agua permiti\'o conocer y almacenar los datos recolectados de los sitios remotos en tiempo real, considerando la sincronizaci\'on entre el tiempo de sleep de los m\'odulos y el tiempo de respuesta de los sensores, \cite{samaniegodevelopment}. 
Torres, D. manifiesta que estudiar las t\'ecnicas de medici\'on de los par\'ametros de calidad de agua y las variaciones de los mismos, permitieron que el sistema electr\'onico dise\~nado se tomar\'a como modelo para la medici\'on de los par\'ametros de calidad de agua en cualquier punto sobre el río Cauca del Valle Del Cauca (Colombia), \cite{torres2009diseno}.
R.G. Jones, ha desarrollado un sistema de referencia para la medici\'on de la conductividad del agua, el m\'etodo se basa en la medici\'on de la resistencia de una columna de agua de dimensiones conocidas con precisión. Hay un efecto de polarización del electrodo y la convención es extrapolar la conductividad en funci\'on de la frecuencia inversa para encontrar el valor en frecuencia inversa cero. Las mediciones pueden realizarse con una incertidumbre de aproximadamente 0,14, \cite{jones2002measurements}v.
Casper y otros, realizaron el seguimiento y la evaluaci\'on, especialmente la identificaci\'on de patrones o tendencias espaciales en la qu\'imica del agua (temperatura, conductividad, salinidad, turbidez, clorofila, materia org\'anica disuelta y los gases disueltos) con el uso de una innovadora combinaci\'on de veh\'iculos no tripulados de superficie (USV) y t\'ecnicas de an\'alisis geoespaciales a modo de demostrar que la percepci\'on de que los par\'ametros de muestreo y an\'alisis a intervalos regularmente espaciados sobre la superficie de un sistema fluvial ser\'an representativos de las tendencias generales, sin embargo, la estrategia de muestreo est\'andar asume tanto que los par\'ametros cambiar\'a de una manera longitudinal y aguas abajo y que la media de un par\'ametro es el nivel donde se producen impactos negativos, \cite{casper2007combining}. 
Li, Meilin y otros, presentan en este trabajo, el dise\~no y la implementaci\'on de un nuevo sistema USV reconstruido desde una lancha para proporcionar comodidad para los experimentos de control de aut\'onomas de futuros. El nuevo sistema USV dise\~nado se compone de tres partes: cuerpo principal, el sistema de control de ordenador de a bordo y el sistema de control de tierra, \cite{li2012design}. 
Mastmija y otros, proponen, en este trabajo, el desarrollo de un sistema de control para un veh\'iculo submarino autónomo dedicado a la observaci\'on de los oc\'eanos. El veh\'iculo, un h\'ibrido entre veh\'iculos submarinos aut\'onomos (AUV) y Veh\'iculos de superficie Aut\'onomo (ASV), se mueve sobre la superficie del mar y hace inmersiones verticales para obtener perfiles de una columna de agua, de acuerdo con un plan preestablecido. El desplazamiento del veh\'iculo en la superficie permite la navegaci\'on por GPS y la comunicaci\'on de telemetr\'ia por radio-modem. El sistema de control est\'a basado en un equipo integrado, est\'a dise\~nado y desarrollado para este veh\'iculo que permite la navegaci\'n aut\'onoma de un veh\'iculo. Este sistema de control se ha dividido en subsistemas de navegaci\'on, propulsi\'on, de seguridad y de adquisici\'on de datos. Dado su alto rendimiento, la incorporaci\'on de algoritmos de control de trayectoria es factible. Tambi\'en, hardware y software espec\'ificos dise\~nados para el correcto funcionamiento de los sensores y propulsores, \cite{masmitja2010development}.
Andrew J. y otros, han desarrollado un sensor de temperatura de bajo costo que puede ser configurado en "cadenas" sumergibles distribuyé\'endolos a lo largo de un simple cable de tres hilos. El sensor ha demostrado ser capaz de entregar las mediciones de temperatura altamente coincidentes simult\'aneas a una resoluci\'on de unas pocas mil\'esimas de un grado, con una coincidencia mejor que 0.01  \textsuperscript{o}C y una incertidumbre de la medici\'on de aproximadamente 0,05  \textsuperscript{o}C. Una t\'ecnica para generar un recuento estandarizado frente a la curva de temperatura se ha desarrollado utilizando el m\'etodo de diferencias finitas, \cite{skinner2006using}.
Borden  y otros ,  describieron en este documento un m\'etodo para emplear m\'odems ac\'usticos subacu\'aticos para el intercambio de datos e informaci\'on  de control entre un AUV sumergida y el operador. Ejemplos de datos que ser\'ian de  inter\'es para un operador AUV podr\'ian consistir en: estado de carga de la bater\'ia, el rumbo del veh\'iculo, la profundidad del veh\'iculo, y la distancia acumulada recorrida. Este  tipo de datos puede ser transferido a través de comunicaciones acústicas subacu\'aticas, eliminando la  necesidad de que el veh\'iculo a la superficie de forma rutinaria y transmitir estos datos a trav\'es del aire. M\'odems ac\'usticos subacuáticos se pueden emplear de baja frecuencia (9 a 13 kHz) ac\'ustica para lograr  una comunicación efectiva. Las distancias de transmisi\'on de m\'as de 5000 metros se  pueden lograr con frecuencias en este rango, \cite{borden2012long}.
P.Ramos y otros  presentan un nuevo sensor de cuatro electrodos para mediciones de conductividad del agua. Adem\'as del sensor en s\'i, todo el acondicionamiento de la se\~nal se implementa junto con el procesamiento de la se\'~nal de las salidas del sensor para determinar la conductividad del agua. El sensor est\'a dise\~nado para mediciones de conductividad en el rango de 50 mS / m hasta 5 S / m a trav\'es de la colocaci\'on correcta de los cuatro electrodos dentro del tubo por donde fluye el agua. El prototipo implementado es capaz de suministrar al sensor la corriente necesaria a la frecuencia de medici\'on, adquiriendo las señales sinusoidales a trav\'es de los electrodos de voltaje del sensor y a trav\'es de una impedancia de muestreo para determinar la corriente. Tambi\'en se incluye un sensor de temperatura en el sistema para medir la temperatura del agua y, por lo tanto, compensar la dependencia de la temperatura de la conductividad del agua, \cite{ramos2008four}.
G.Ferri y otros, describen el diseño, el desarrollo y las pruebas en el mar de un novedoso vehículo de superficie autónomo (ASV) de tamaño pequeño diseñado para monitorear la calidad del agua costera. El vehículo se caracteriza por la capacidad de medir concentraciones de hidrocarburos y metales pesados directamente a bordo por medio de sensores miniaturizados hechos a la medida. Esta capacidad, novedosa para un ASV, se combina con un sistema de muestreo basado en cabrestante diseñado específicamente para vehículos de tamaño pequeño. El sistema de muestreo puede recolectar muestras de agua hasta 50 m de profundidad y medir los parámetros físicos/de calidad del agua de la columna de agua. Con estas dos características, HydroNet ASV proporciona un sistema de monitoreo autónomo, práctico y en tiempo real, concebido para complementar las prácticas actuales de monitoreo del agua en las que las muestras son recolectadas por un barco dedicado y analizadas en laboratorios especializados en una etapa posterior,\cite{ferri_hydronet_2015}.

\section{Alcance}
Partiendo de los trabajos de los autores anteriormente mencionados, se establecieron los alcances correspondientes al proyecto, teniendo en cuenta el lugar de implementaci\'on, capacidad logística y financiera, de contar con los equipos o herramientas necesarias.
El TFG comprende el diseño mecánico, eléctrico, electrónico, control y fabricaci\'on  mediantes técnicas de mecanizado de teflón e impresión 3D con poliácido láctico (PLA) de una sonda, con sensores de pH, CE, T, DO, OPR, utilizando usas placas de acondicionamiento off the shelf compatibles con microcontroladores, con autonomía energética propia, capaz de hacer muestreo de fluidos en tiempo real, configurados de forma manual o automático sincronizado con una ruta de algún ASV, almacenando los datos obtenidos en una base de datos local con la posibilidad de transmitirlos a una estación de base remota. 
Para el sensado a multinivel de profundidades, se diseñó e implementó un sistema compuesto por un sonar fijo para detectar la profundidad en el punto de medición, un molinete eléctrico con la capacidad suficiente para soportar el peso de la sonda, 
el sistema de descenso se comunica de forma inalámbrica con la sonda, y se podrá controlar de forma remota, siempre cuando est\'e conectada a una red wifi con internet.