% [RR] FECHA DE HOY jaja 28/09/ 20 :(


\chapter[Introducción]{Introducción}
\pagestyle{fancy}

%\section{Introducción}

El agua dulce es el recurso más importante para la humanidad, indispensable de todas las actividades sociales, económicas y ambientales. 
Es una sustancia necesaria para la vida en nuestro planeta. 
Se lo puede encontrar en la naturaleza en sus tres estados (sólido, liquido y gaseoso) y conforma la  mayor parte de la superficie del planeta. 
Desde el inicio de los tiempos estuvimos en contacto con el agua, como fuente de hidratación, medio de transporte, agricultura, ganadería, materia prima, etc., con el correr de los tiempos pudimos observar como el agua ha sufrido una serie de alteraciones debido a la explotación abusiva por estas actividades. 
Hace relativamente poco tiempo le hemos dado la importancia para estudiar la forma como impactamos y como altera el ecosistema natural, regular varios aspectos con tratados nacionales, internacionales, con el fin de preservar y no alterar mucho los ecosistemas naturales. %Proceso

%[RR]Esta primera parte lo probable es que cambie, con cosas del cap2, le agregue temporalmente el conector para mejorar su nexo 

La humanidad ha sido testigo de como se  perdieron  ecosistemas completos, extinguiéndose fauna, flora, aparici\'on, enfermedades que diezmaron poblaciones, a causa de la contaminaci\'on generada, que alteraban los ecosistemas, y esto modificaba la calidad de los recursos h\'idrico, causando da\~nos al medio ambiente y una serie de consecuencias muchas veces irreversibles. 

El Paraguay es un pa\'is rico en recursos naturales, está ubicado en el número 30, con unos 336 km$^3$ de agua dulce dentro de su territorio. 
El acuífero Guaraní es la tercera mayor reserva de agua dulce del mundo, compartido con Brasil, Argentina y Uruguay, con 70.000 km$^{2}$ de superficie en Paraguay~\cite{stp_Agua}. 
Por ello es de vital importancia que se tengan políticas publicas y departamentos que velen por esos recursos.
Es esencial el monitorio de los recursos hídricos para así detectar cambios y tomar las acciones requeridas a tiempo.

En el país, actualmente existen una gran cantidad de fábricas e industrias que expulsan sus desechos residuales a los afluentes sin previos trabajos de depuración en plantas de tratamientos u otros métodos, que no son controlados y regulados por un gran déficit de fiscalización ambiental. 


%
%
%PENDIENTE Hablar de sistemas de medición de de calidad del agua. PENDIENTE
% [ML] OKI
% [RR] Creo que esto voy a hablar mejor en el cap 2
%

El presente trabajo final de grado (TFG) proponer el diseño y fabricación de una sonda, aplicado al monitoreo de calidad del agua mediante sensores, para la medición automática del potencial del hidrógeno (PH), temperatura (T), oxígeno disuelto(DO), potencial de oxido reducción (OPR), conductividad eléctrica (CE). 
Los cuales según estándares nacionales e internaciones son indicadores básicos para analizar el estado del recurso hídrico y poder inferir sobre la calidad del mismo.  


%Hablar del proyecto PINV15-177 
El presente TFG se enmarca en el proyecto PINV 15-177 - Vehículo Autónomo de Superficie (ASV) para el estudio de calidad del agua en lagos y lagunas, desarrollado en el Laboratorio de Sistemas Distribuido (LSD) perteneciente a la Facultad de Ingeniería de una Universidad Nacional de Asunción (FIUNA) con sede en el Centro de Innovación Tecnológica (CITEC). 
El ASV proporciona una automatización para el muestreo, transportando la sonda hasta la locación deseada, georeferenciar los datos obtenidos y capacidad de sensado a diferentes niveles de profundidad. 
Como mencionan \cite{guiaHANNA}en su gu\'ia, sobre es importante reconocer que la calidad del agua puede verse afectada de manera negativa por razones naturales y  además por factores humanos. y que monitoreando de manera regular las fuentes de agua puede ayudar a identificar problemas potenciales antes de que generen algún daño.


El TFG  se organiza en cinco capítulos, desarrollados de la siguiente manera. 
El Capítulo uno, se presentará los objetivos generales y específicos del TFG,  y breve an\'alisis de la problemática y se establece el alcance del trabajo realizado.
El Capítulo dos, describe los principales antecedentes sobre sobre el monitoreo de recursos hídricos, además de las definiciones relevantes.
El Capítulo tres, se desarrolla el diseño de las parte, lo concerniente al desarrollo del sistema y la integración de todos sus componentes.   
El Capítulo cuatro, se desarrolla las pruebas realizadas y resultados obtenidos con la sonda. 
Capítulo cinco,  presentará las conclusiones obtenidas y recomendaciones.
%[RR] Yo metí todo dentro de "sistema " realmente , podria agregar una descripción mas detallada para que se entienda lo que abarca  

\section{Objetivo General}
Diseñar e implementar una sonda y sistema de despliegue para medición a multiniveles de factores fisicoquímicos de recursos hídricos.


\section{Objetivos Específicos}
\begin{itemize}
	\item Diseñar la pieza mecánica de la sonda que contendrá a los sensores de CE, pH, OD, OPR, T.
    \item Investigar y fabricar  la sonda en el material apto.
    % [RR] Yo primero hice el diseño para impresora 3D y después vi como fabricar, onda lo que queria es el motivo de la impresión 3D y comprar en el mecanizado convencional, yo use ambos 
    \item Realizar pruebas de sondas en ambientes controlados.
    \item Diseñar el sistema de descenso para muestreo a multiniveles.
    \item Diseñar la interfaz gráfica y almacenamiento en base de datos.  
    \item Realizar mediciones fisicoquímicas in situ en los puntos de muestreo del lago Ypakarai.

\end{itemize}

\section{Problemática} 

 Las técnicas tradicionales de muestreo  de aguas  pueden de forma manual, resultar bastante engorrosos,lentos y costoso por los preparativos logísticos anterior y posterior a la campaña de muestro, por lo general se limitan a unas pocas muestras para su posterior análisis. 
 Estos procedimientos muchas veces enlentecen localizar rápidamente de donde provienen los contaminante y conocer le estado real del cuerpo de agua analizado. El acceso a estos datos analizados no son de fácil acceso y están acompañados de burocracia.  


% [RR] Cambie varias cosas,pendiente de mejorar mas explicando los problemas 
 
El presente TFG, pretende entregar al campo de monitoreo y sensado de recursos hídricos, un dispositivo practico, escalable, portátil, inalámbrico, de respuesta rápida, de fácil uso, que almacene los datos recolectado en una base de datos local y pueda transmitir en tiempo real a alguna alguna estación remota, generando automáticamente un reporte una vez concluida el trabajo. 

La sonda diseñada podra ser adaptado para operar con el ASV del proyecto PINV15-177, obteniendo de este sus datos georeferenciales.
Ofrecer\'a una solución de automatizaci\'on del proceso de toma de muestras, optimizando  las campañas de verificaci\'on, con menos log\'istica, menos costosa y m\'as r\'apidas,  mucho m\'as puntos de muestreo y generaci\'on autom\'atica de reporte cuando sea necesario.

%[RR] Reescribí todo 

\section{Estado del Arte}
Aqu\'i podría hablar del desarrollo de otros sistemas de sonda que se encuentran en la

\section{Alcance}
El TFG comprende el diseño mecánico, eléctrico, electrónico, control y fabricación de una sonda, que tenga incorporados los sensores de pH, CE,T, DO, OPR, con autonomía energética propia, capaz de hacer muestreo de fluidos en tiempo real, configurados de forma manual o automático sincronizado con una ruta de algún ASV, almacenando los datos obtenidos en una base de datos local con la posibilidad de transmitirlos a una estación de base remota. 

Para el sensado a multiniveles de profundidades, se diseñó e implementó un sistema compuesto por dos sonares, uno fijo para detectar la profundidad en el punto de medición y otro móvil como detector de obstáculos, un molinete eléctrico con la capacidad suficiente para soportar el peso de la sonda.

El sistema de descenso se comunican de forma inalámbrica con la sonda, y se pueden controlar de forma remota, siempre cuando este conectada a una red wifi con internet.

El modo de operación podría ser automático o manual, predefinido por el usuario.

%[RR]Cambie varias cosas 



%\section{Estado del arte}

%[RR] Me gustaría cerrar por de pronto este capitulo, es justo eso como mencionabas son las referencias para cuando empecé a trabajar, no tienen contexto entre sí, y supongo de ni deberían jaja, yo hice copy paste para tener más a mano cuando quiera usarlos, voy a hacer las correcciones que me sugerís . Arigato  .
