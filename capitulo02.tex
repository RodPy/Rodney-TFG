

\chapter[Capítulo 2. Marco Te\'orico.]{Marco Te\'orico\label{etiqueta}}
%-------------------------------------------------------------------------------------
% [RR] En este cap quiero iniciar con empezar hablando un poco de lo que es calidad del agua, hablar sobre los factores físico - químicos quede deben de tenerse en cuenta, profundizar los parámetros físicos químicos que yo utilizo (temp, ph, conductividad eléctrica, opr, salinidad, total solidos disueltos, oxigeno disuelto.  .
% [RR]Luego quiero profundizar en cada sección:
% [RR]-> un poco sobre cada uno de los sensores, tipo, funcionamiento, 
% [RR]-> Tipos de muestreos manuales y automáticos, opciones comerciales, (énfasis al valor agregado de medicion a multinivel)


\section{Monitoreo de Agua .}
El agua, un recurso presente en la naturaleza, indispensable para la vida misma e infiere en la mayoría de los procesos industriales[ver ref], en el planeta el 2.5 \% del agua es dulce de los cuales el 69 \% se encuentra en estado solido en los polos y cumbres de las monta\~nas altas del alrededor del mundo  [verRef].
Se estima que un XX\% de los recursos h\'idricos alrededor del mundo están contaminados por diferentes, haciendo que estos los ecosistemas el cual conforman se da\~nen gravemente y si es que no se toman las medidas de mitigaci\'on a tiempo es posible que los da\~nos se tornen irreversible, alterando gravemente al medio ambiente que a su vez un sin fin de problemas relacionados como, perdida de fauna, flora y paisaje, enfermedades a las poblaciones alera\~nas adem\'as de todo el impacto econ\'omico y social que conlleva.

Muchas veces es complicado conocer el estado real de un recursos h\'idrico, donde uno de los factores que influye la t\'ecnica de monitoreo, ya sea de forma manual o por sensores fijos, el monitoreo manual involucra un despliegue log\'istico importante de los t\'ecnicos, especialistas, materiales y transporte para llegar a los puntos de medici\'on, que muchas veces se dificulta por entorno. % Nota: pendiente mejorar, onda que una lancha no llega a todas partes, que no puede navegar por varios dias, etc .
Las sondas fijas no requieren de constantes despligues log\'isticos, posteriores a su  instalaci\'on, pero si requiere de un inversi\'on mucho m\'as considerable y peri\'odicos mantenimientos, por su alto costo de instalaci\'on no es posible cubrir en su totalidad del recurso h/'idrico a ser analizado. % Consultar al PTI costo para análisis económico

El Paraguay, es un pa\'is muy rico hidro\'ogicamente, por la cantidad de agua dulce en su territorio segun XXX contamos con XX rios, XX lagunas, xx lagos, XX acuiferos, XX estero, xx blabla, distribuidos en todo el territorio, donde se torna una gran tarea de log\'istica las expediciones de monitorio y econ\'omicamente inviables dotar de sensores fijos de monitoreo, donde un sistema de monitoreo m\'as vers\'atil podr\'ia optimizar el trabajo y 


En Paraguay el Decreto Reglamentario 222/02 (Ley 14.859, Uruguay) establece valores límites para determinados parámetros y representa la referencia de calidad del agua a nivel nacional. Este decreto clasifica el estado de los cuerpos de agua continentales en 4 clases:

Clase 1
Aguas destinadas o que puedan ser destinadas al abastecimiento de agua potable a poblaciones con tratamiento convencional.

Clase 2
a. Aguas destinadas al riego de hortalizas o plantas frutícolas u otros cultivos destinados al consumo humano en su forma natural, cuando éstas son usadas a través de sistemas de riego que provocan el mojado del producto.
b. Aguas destinadas a recreación por contacto directo con el cuerpo humano.

Clase 3
Aguas destinadas a la preservación de los peces en general y de otros integrantes de la flora y fauna hídrica, o también aguas destinadas al riego de cultivos cuyo producto no se consume en forma natural o en aquellos casos que siendo consumidos en forma natural se apliquen sistemas de riego que no provocan el mojado del producto.

Clase 4
Aguas correspondientes a los cursos o tramos de cursos que atraviesan zonas urbanas o suburbanas que deban mantener una armonía con el medio, o también aguas destinadas al riego de cultivos cuyos productos no son destinados al consumo humano en ninguna forma.

Cabe señalar que medidas de un parámetro o de algunos parámetros que superen los valores estándar establecidos en el Decreto 253/79 no necesariamente indican una mala calidad del agua. Para conocer con mayor precisión el estado del agua es necesario considerar el conjunto de los parámetros asociado a la información recogida a través de otros elementos físicos y biológicos del ecosistema acuático. También es importante resaltar que hay propuestas para modificar los valores límites que se disponen en el decreto (Deci Agua, 2016)


\section{opop}

Monitoreo del agua potable : los parámetros químicos comunes incluyen pH, nitratos y oxígeno disuelto. La medición de O2 (o DO) es un indicador importante de la calidad del agua. Los cambios en los niveles de oxígeno disuelto indican la presencia de microorganismos de aguas residuales, escorrentías urbanas o agrícolas o descargas de fábricas. Un nivel adecuado de ORP minimiza la presencia de microorganismos como  E. coli, Salmonella, Listeria . Los niveles de turbidez por debajo de 1 NTU indican la pureza adecuada del agua potable.
Detección de fugas químicas en ríos : pH extremo o valores bajos de OD indican derrames químicos debido a problemas en la planta de tratamiento de aguas residuales o en la tubería de suministro.
Medición remota de piscinas : la medición del potencial de oxidación-reducción (ORP), el pH y los niveles de cloro del agua puede determinar si la calidad del agua en piscinas y spas es suficiente para fines recreativos.
Niveles de contaminación en el mar : la medición de los niveles de temperatura, salinidad, pH, oxígeno y nitratos proporciona información para los sistemas de detección de calidad en el agua de mar.
Prevención de la corrosión y depósitos de cal:  Controlando la dureza del agua podemos evitar la corrosión y depósitos de cal en lavavajillas y dispositivos de tratamiento de agua como calentadores. La dureza del agua depende de: pH, temperatura, conductividad y concentraciones de calcio (Ca + ) / magnesio (Mg 2+ ).
Cultivo de abetos / Monitoreo de tanques de peces / Criadero / Acuicultura / Acuaponía: Medición de las condiciones del agua de animales acuáticos como caracoles, peces, cangrejos de río, camarones o langostinos en tanques. Los valores importantes son el pH, el oxígeno disuelto (DO), el amoníaco (NH 4 ), el nitrato (NO 3 - ), el nitrito (NO 2 - ) y la temperatura del agua.
Hidroponía : las plantas que toman los nutrientes directamente del agua necesitan un pH preciso y niveles de oxígeno en el agua (OD) para obtener el máximo crecimiento.

Las sondas del sensor miden más de 12 parámetros químicos y físicos de la calidad del agua, como pH, nitratos (NO3), iones disueltos (fluoruro (F - ), calcio (Ca 2+ ), nitrato (NO 3 - ), cloruro (Cl - ), Yoduro (I - ), Cúprico (Cu 2+ ), Bromuro (Br - ), Plata (Ag + ), Fluoroborato (BF 4 - ), Amoníaco (NH 4 ), Litio (Li + ), Magnesio (Mg 2+ ) , Nitrito (NO 2- ), Perclorato (ClO 4 ), Potasio (K + ), Sodio (Na +) oxígeno disuelto (OD), conductividad (salinidad), potencial de oxidación-reducción (ORP), turbidez, temperatura, etc. Los contaminantes pueden detectarse y tratarse en tiempo real para garantizar una buena calidad del agua en toda la red de suministro de agua. Los valores extremos de pH pueden indicar derrames de productos químicos, problemas en la planta de tratamiento o problemas en las tuberías de suministro. Los niveles bajos de OD pueden indicar la presencia de microorganismos debido a escorrentías urbanas / agrícolas o derrames de aguas residuales. El ORP mide qué tan bien está funcionando la desinfección del agua.

\section{Calidad del agua.}
%Tema pendiente de ver 
%Preguntar en http://habitat.aq.upm.es/dubai/00/bp561.html sobre los parametros y estandares Py.
%%%%%%%%%

El término calidad del agua se utiliza para describir el estado del agua, incluida su características químicas, físicas y biológicas, generalmente con respecto a su idoneidad para un propósito particular (es decir, beber, nadar o pescar). \cite{waterquality}

La calidad del agua se puede considerar como una medida de la idoneidad del agua para un uso particular en función de determinadas características físicas, químicas y biológicas. Para determinar la calidad del agua, los técnicos primero miden y analizan las características del agua, como la temperatura, el contenido de minerales disueltos y la cantidad de bacterias. Luego, las características seleccionadas se comparan con normas y pautas numéricas para decidir si el agua es adecuada para un uso particular. La calidad es el estado de las características físicas, biológicas y químicas del agua basados en condiciones establecidas

Los par\'ametros \'optimos de las caracter\'isticas f\'isicas, qu\'imicas y biolog\'ias del agua, se establecen seg\'un las normas de los organismos reluladores. En Paraguay el organismo regulador 
% [RR] Concluir esta sección con los parámetros de algún organismo 

\section{Sensores para calidad de agua.}
Existen varios parámetros que pueden estudiarse para indicar la calidad del agua. Estos parámetros pueden medirse ya sea de características físicas como el pH, conductividad, o temperatura; niveles de varios nutrientes en agua, como nitratos y fosfatos; o de algunos elementos y compuestos del agua, como el oxígeno disuelto. 

Existen una gran variedad de sensores para la obtenci\'on de las mediciones de los \'indices de calidad del agua, en el presente TFG abordaremos los sensores de pH, conductividad el\'ectrica, oxigeno disuelto, potencial de oxido reducci\'on y temperatura. 

%

pH 
¿Qué es el pH?
El pH es la medición de la concentración relativa de iones hidronio e hidroxilo en el agua. La escala de rangos va desde 0 a 14, con 0 siendo una solución de ácido fuerte y 14 una solución básica fuerte.

Escala pH

¿Porque la medición de pH es importante?
El pH es una manera de evaluar que tan adecuada es el agua para una planta o animal.  Si el agua es demasiado ácida o básica, ya sea a por contaminantes naturales o de origen humano, puede ver un impacto profundamente negativo a la vida acuática.  Un pH normal en cuerpos de agua tiene un valor de entre 5.0 a 9.0, pero de manera ideal debería estar en un rango de entre 6.0 a 8.0.

¿Cómo medir el pH?
Test de pH comunes cómo los test kit químicos y tiras de tornasol, son simples y económicos. Aun así, estos métodos tienen inconvenientes que pueden causar resultados imprecisos. Ambos métodos entregan un valor de pH basado en una reacción química que resulta en el cambio de color. Cuando su papel o muestra líquida cambia de color, debe compararlo con la guía de color y allí seleccionar la que más se ajuste según su criterio.

Una lectura más precisa significa que es necesario utilizar un medidor de pH. Cuando está eligiendo un tester o medidor de pH, existen múltiples consideraciones relacionadas tanto al electrodo como al dispositivo que deben tenerse en cuenta. Asegúrese de encontrar un medidor de pH y un electrodo que mejor se ajuste a su área de trabajo.

Calibración de pH en campo
Lo primero a tener en cuenta es elegir el valor de las soluciones buffer. Estas deben tener una pendiente que se ajuste a su valor esperado de pH. ¿Porque utilizar varios buffer? La calibración en dos o más puntos, consiste en calibrar el medidor en un punto superior y otro inferior al rango de pH esperado, agregando más puntos para mejorar el ajuste de la pendiente. Por ejemplo, si desea medir el pH del jugo de limón, que de manera habitual tiene un pH de 2, puede utilizar buffers técnicos 1.00 y 4.01 para la calibración en dos puntos. Si no conoce el pH en sus muestras de agua, entonces es necesario un tercer punto de calibración para asegurar una mayor precisión.

Procedimiento de calibración:

Llene el beaker con suficiente buffer de calibración de pH para cubrir la unión del electrodo (cerca de 75 mL de un beaker de 100 mL)
Ubique el electrodo en el beaker que contiene el buffer de solución de calibración de pH y agite suavemente.
Confirme el punto de calibración cuando la lectura sea estable, o cuando los dígitos no cambien por al menos 5 segundos.
Repita para puntos de calibración adicionales. Asegúrese de enjuagar con agua pura entre los puntos de calibración. Se recomienda realizar una calibración en al menos dos puntos.
La calibración de pH se completó. Enjuague la sonda con agua desionizada y almacene la sonda de acuerdo a las instrucciones del fabricante.

Tulipanes


Temperatura
¿Qué es temperatura?
La temperatura es una de las medidas más comunes en la vida diaria. En el contexto de calidad de agua, la temperatura puede proveer un indicio de las condiciones de vida para plantas acuáticas y animales.  Las temperaturas templadas se consideran generalmente benéficas para el crecimiento de la población acuática. De cualquier forma, después de cierto punto la temperatura puede tener un efecto contrario, contribuyendo a declinar la diversidad biológica en cuerpos de agua.

¿Porque la medición de temperatura es importante?
Organismos acuáticos cómo los peces y plancton son de agua fría, así que la temperatura en el agua tiene un impacto directo en la temperatura de su cuerpo. Estos organismos tienen rangos de temperatura en los cuales pueden sobrevivir y desarrollarse. A medida que la temperatura alcanza el límite superior o su rango para un organismo, la actividad biológica estará en su tope. La actividad disminuirá cuando se alcance el punto mínimo del rango. Si la temperatura excede el rango aceptable para el organismo, la cantidad disponible de oxígeno puede ser demasiado baja para sustentarlos. Esto se debe a que el agua caliente tiene un punto de saturación de oxigeno menor al agua fría. Si la temperatura está por debajo del rango aceptable, no hay suficiente actividad para el crecimiento de los organismos. La temperatura alta también contribuye al crecimiento y floración de algas. El oxígeno se consume en la medida que bacterias descompongan estos brotes, lo que reduce la cantidad de oxígeno disuelto disponible.
La temperatura en varios cuerpos de agua se basa en la hora del día y la cantidad de luz solar calentando la superficie. Las temperaturas aceptables también varían dependiendo del tipo de rio o corriente de agua que desee monitorear. Esto depende de la fuente de la cuenca que alimenta la corriente. Si se alimenta la corriente con agua de manantial, por ejemplo, la temperatura normal de la corriente puede ser adecuada (menor a 68°F). Una corriente se considera templada si tiene una temperatura promedio superior a los 68°F pero menor a los 89°F. La temperatura también puede verse influenciada por el flujo y el cuerpo de agua. Si el flujo de agua se incrementa, quizá cómo resultado de una fuerte lluvia, se puede esperar que la temperatura disminuya. El incremento en la corriente tiene cómo efecto reducir la temperatura en el agua.

La polución por temperatura, también conocida cómo polución térmica, puede ser causada por vertimientos de agua calentada en asfalto o concreto. Esta también puede proceder de efluentes industriales que sean descargados en cuerpos de agua, o agua que sea usada cómo refrigerante en plantas de energía nuclear. Estos efluentes significan que el agua en el que son descargadas incrementará la temperatura general del cuerpo de agua.  La temperatura también puede asociarse a la turbidez.  Ya que la cantidad de luz absorbida incrementa a medida que el agua se oscurece, la temperatura también aumentará.

¿Cómo medir la temperatura?
Muchos termómetros simples usan un termistor.  El termistor es un dispositivo semiconductor cuya resistencia varía en función de la temperatura.  A medida que la temperatura incrementa, la resistencia disminuye.  La resistencia medida por el termistor se convierte en un valor que se muestra ya sea en la escala Celsius o Fahrenheit.  Los sensores termistores son adecuados para rangos de temperatura desde -50° a 150°C (-58° a 302°F).

Calibración de temperatura
Muchos medidores se calibran en fábrica para las lecturas de temperatura. Es una buena práctica revisar esta calibración al menos una vez al año en un laboratorio, y asegurar que el sensor de temperatura funciona de manera adecuada.

Conductividad (CE) / Total de sólidos disueltos (TDS)
¿Qué es la conductividad?
La conductividad eléctrica (CE) mide que tan bien una sustancia puede transmitir una corriente eléctrica. Pequeñas partículas cargadas, llamadas iones, pueden ayudar a transportar la corriente eléctrica a través de la substancia. Estos iones pueden estar cargados positiva, o negativamente. A mayor cantidad de iones disponibles, mayor será la conductividad; menores iones resultarán en una menor conductividad. La CE se reporta de manera habitual en milliSimens por centímetro (mS/cm).

El total de sólidos disueltos (TDS) es la cantidad de sustancias disueltas en soluciones. Las mediciones permiten conocer las sustancias orgánicas e inorgánicas disueltas en el líquido. Los resultados de esta lectura se muestran en miligramos por litro (mg/L), partes por millón (ppm), gramos por litro (g/L), o partes por mil (ppt).

¿Porque la medición de conductividad es importante?
La conductividad eléctrica (CE) es otra manera de evaluar la calidad del agua, ya que al incrementar la presencia de total de sólidos disueltos (TDS), expresada en la CE, puede ser un indicador de contaminantes. La CE puede verse afectada por los carbonatos presentes en el agua caliza, contaminantes humanos como aguas residuales u otro tipo de fuente cómo sistemas de pozos sépticos o residuos de agricultura.

Altas concentraciones de TDS pueden reducir la calidad de agua y causar problemas en el balance de agua para organismos individuales. Por otra parte, bajas concentraciones pueden limitar el crecimiento de vida acuática. Algunos de los efectos comentados para parámetros como el dióxido de carbono y acidez tienen relevancia para la CE, cómo su impacto negativo en la fotosíntesis. Esto se debe a que al incrementar la cantidad de sólidos el agua se oscurecerá, reduciendo la tasa de fotosíntesis. La CE proveerá un indicio del total de sólidos disueltos, del cual el total de sales disueltas es un componente. Si el nivel de sales en el TDS es alto, esto también puede contribuir a acidificar el agua. De cualquier forma, si el nivel de carbonatos en la lectura de TDS son altos, esto podría contribuir a incrementar la alcalinidad, lo que puede ayudar a proteger el agua ante cambios ácidos. Esta es una buena respuesta de interrelación, entre los parámetros de calidad de agua.

Niveles aceptables de CE en ríos y corrientes de agua varía dependiendo del tipo de sólidos disueltos presentes y esto determina el uso de la corriente, ya sea para pesca, nado o como una fuente de agua potable.


Corriente


Es importante comprender la relación entre TDS y sólidos totales. Los sólidos totales se refieren a toda la materia sólida, ya sea suspendida o disuelta en agua. Los sólidos disueltos no son visibles en el agua, ya que al disolverse se vuelven parte de la solución. Los TDS son una medida de las sustancias disueltas en agua que están en una muestra de agua. En una muestra recolectada en un rio, estas sustancias disueltas se conocen cómo solutos, y el agua es llamada solvente.

¿Cómo medir la conductividad?
La mejor manera de medir la conductividad es con el uso de un medidor. Dos electrodos que aplican un voltaje AC se ubican en la solución. Esto crea una corriente que depende de la conductividad natural de la solución. El medidor lee esta corriente y muestra la conductividad (CE) o ppm (TDS)

Calibración de conductividad en campo
Es importante calibrar la conductividad antes de analizar la muestra. Esto se debe a que recubrimientos aceitosos y contaminantes biológicos puede cambiar la geometría celular, resultando en una desviación en la constante celular. Antes de realizar la calibración de conductividad, siempre inspeccione el sensor de CE por residuos u obstrucciones.

La mayoría de los medidores se calibran con un solo estándar, procurando que este cerca de la conductividad de la muestra a medir. Un segundo estándar puede usarse para revisar la linealidad del instrumento en el rango de medición.

Procedimiento de calibración:

Llene el beaker con suficiente estándar para cubrir la unión del electrodo (cerca de 75 mL de un beaker de 100 mL) Vierta solución adicional en un segundo beaker para enjuagar el sensor.
Ubique el electrodo en el beaker de enjuague y asegúrese de que los canales del sensor de CE estén llenos con estándar fresco al enjuagar y agitar la sonda en el beaker unas cuantas veces.
Ubique la sonda en el beaker de calibración y golpee suavemente para liberar burbujas atrapadas.
Confirme el punto de calibración cuando la lectura sea estable, o cuando los dígitos no cambien por al menos 5 segundos. (Algunos medidores requerirán que ingrese el valor del estándar de conductividad)
La calibración esta completa. Enjuague la sonda con agua desionizada y almacénela de acuerdo a las instrucciones del fabricante.
Oxígeno disuelto (OD)
¿Qué es el oxígeno disuelto?
La concentración de oxígeno disuelto (OD) en agua es extremadamente importante en la naturaleza, al igual que para los ambientes artificiales.  En océanos, lagos, ríos, y otros grandes cuerpos de agua, el oxígeno disuelto es esencial para el crecimiento y desarrollo de la vida acuática.  Sin oxígeno, el agua puede volverse tóxica debido al decaimiento de la materia orgánica por bacterias anaeróbicas.  En un ambiente industrial, el agua debe contener al menos 2 mg/L de oxígeno para proteger a las tuberías de corrosión.  De cualquier manera, los sistemas de agua de calderas, en muchos casos no pueden contener más que 10 mg/L de oxígeno disuelto.

¿Porque es importante el oxígeno disuelto?
Los niveles de OD pueden ayudar a indicar la salud de un cuerpo de agua. Si los niveles de OD están dentro de la media o más altos, el agua es un buen ambiente para una amplia variedad de vida acuática. Si los niveles de OD son bajos, indica la presencia de contaminantes en el agua. Algunas especies de vida acuática pueden subsistir en agua con un amplio rango de OD; pero otras fallecen a bajos niveles de OD.


Oxígeno disuelto


Se espera que las lecturas de OD tengan una amplia fluctuación si la fuente de agua cuenta con abundante vida vegetal. Esto se debe al proceso de fotosíntesis. Ya que existe una menor actividad fotosintética en las noches, cuando la luz no está presente, y que tanto plantas como animales seguirán consumiendo oxígeno a través de la respiración, los niveles de OD en las horas de la mañana serán mucho menores a los de otras horas del día. Una vez inicia la fotosíntesis, los niveles de OD incrementarán. Este es un buen ejemplo de los beneficios de medir los parámetros varias veces durante el día. Si únicamente se realiza la medición de OD antes del amanecer, se llegará a una conclusión inadecuada sin importar la salud del cuerpo de agua.

Mientras que los niveles de OD se influencian particularmente por la actividad fotosintética, una gran cantidad de OD se obtiene de la mezcla del OD y el agua. Esto sucede si el agua es turbulenta en grandes cuerpos de agua. La turbulencia incrementa el área superficial del agua, para que el oxígeno atmosférico puede mezclarse más fácilmente. El aire tiene una concentración de oxígeno que es 20 veces mayor que la concentración de oxígeno en el agua. La diferencia de concentración resulta en el oxígeno atmosférico disolviéndose en agua cuando las dos se encuentran. Si hay más superficie de agua en su interfaz, entonces más oxígeno del aire se absorberá.

Otros factores que influencian los niveles de OD son la temperatura y los vertimientos. El oxígeno se disuelve más fácilmente en agua fría, y el agua fría tiene la capacidad de mantener mayores cantidades de gas que el agua tibia, así que el nivel de OD disuelto disminuye a medida que el agua se calienta. Los vertimientos pueden incluir desechos orgánicos o contaminantes creados por el hombre; en ambos casos, los organismos en el agua deben usar oxígeno en el proceso de descomponer estos contaminantes.  También, los desechos orgánicos pueden llevar al crecimiento de vegetación acuática.  Cuando las plantas mueren al final del lapso de crecimiento, grandes cantidades de oxígeno disuelto se consumen al descomponerse.

¿Cómo podemos medir el oxígeno disuelto?
Las concentración de oxígeno disuelto se reporta en miligramos de gas por litro de agua, mg/L. (La unidad equivalente a mg/L es equivalente a partes por millón=ppm)  Las lecturas de OD se realizan habitualmente utilizando una sonda y un medidor.

 
Es importante realizar lecturas de OD en varios momentos del día, y a varias profundidades. Las mediciones le darán una visión general de los niveles de OD en el cuerpo de agua que investiga. Cómo sucede con todos los parámetros de calidad de agua, este parámetro debe monitorearse en el tiempo. Esto producirá suficiente información para identificar y evaluar tendencias.

Calibración de OD en campo
El contenido de oxígeno disuelto (OD) en el agua se mide usando un electrodo con membrana. Desafortunadamente, cepillos u otros objetos de limpieza pueden dañar la membrana, así que remplazar la tapa de la membrana y el electrolito es la mejor manera de realizar un mantenimiento periódico. Si bien es más fácil calibrar el sensor de OD antes de estar en campo, es mejor calibrar la sonda en campo pues las diferencias de altitud y presión barométrica entre la calibración y la lectura pueden resultar en errores. Asegúrese de verificar que las lecturas de presión barométrica, conductividad, y temperatura sean correctas.

Procedimiento de calibración (100%):

Llene un beaker de calibración con agua (como alternativa ubique una esponja húmeda o una toalla de papel húmeda en el fondo del contenedor usado para la calibración)
Ajuste ligeramente la sonda en el beaker de calibración para evitar escape de humedad. *Asegúrese de no humedecer su sensor de OD por la evaporación en el sensor de temperatura pues esto puede influenciar las lecturas de la sonda durante la calibración.
Permita al contenedor saturarse con vapor de agua (por aproximadamente 10 a 15 minutos). *Durante este periodo, encienda el instrumento para permitir que la sonda de O.D. caliente
Confirme el punto de calibración cuando la lectura sea estable, o cuando los dígitos no cambien por al menos 5 segundos.
La calibración esta completa. Enjuague la sonda con agua desionizada y almacénela de acuerdo a las instrucciones del fabricante.
Procedimiento de calibración (0%):

Llene el beaker con suficiente solución 0% OD para cubrir la unión del electrodo (cerca de 75 mL de un beaker de 100 mL)
Sumerja la sonda de OD en la solución.
Confirme el punto de calibración cuando la lectura sea estable, o cuando los dígitos no cambien por al menos 5 segundos.
La calibración esta completa. Enjuague la sonda con agua desionizada y almacénela de acuerdo a las instrucciones del fabricante. *Asegúrese de enjuagar toda la solución 0% OD para que esta no afecte las mediciones de sus muestras.



















%%%%%%%%%%%%%%%%%%%%%%%%%%%%%%%%%%%%%%%%%%%%%%%%%%%%%%%%%%%%%%%%%%%%%%%%%%%%%%%%%%%%%%%%%%%%%%%%%%%%%%%%%%%%%%%%%%%%%%%%%%


\subsubsection{Sensor de potencial de hidrógeno (pH)}

El medidor de pH es un instrumento utilizado para medir la acidez o la alcalinidad de una solución, también llamado de pH. El pH es la unidad de medida que describe el grado de acidez o alcalinidad y es medido en una escala que va de 0 a 14, con 0 siendo una solución de ácido fuerte y 14 una solución básica fuerte.
Es una manera de evaluar que tan adecuada es el agua para una planta o animal.  Si el agua es demasiado ácida o básica, ya sea a por contaminantes naturales o de origen humano, puede ver un impacto profundamente negativo a la vida acuática.  Un pH normal en cuerpos de agua tiene un valor de entre 5.0 a 9.0, pero de manera ideal debería estar en un rango de entre 6.0 a 8.0. %poner refdel 2222

%PONER IMAGEN


¿Cómo medir el pH?
Test de pH comunes cómo los test kit químicos y tiras de tornasol, son simples y económicos. Aun así, estos métodos tienen inconvenientes que pueden causar resultados imprecisos. Ambos métodos entregan un valor de pH basado en una reacción química que resulta en el cambio de color. Cuando su papel o muestra líquida cambia de color, debe compararlo con la guía de color y allí seleccionar la que más se ajuste según su criterio.
Las informaciones cuantitativas dadas por el valor del pH expresan el grado de acidez de un ácido o de una base en términos de la actividad de los iones de hidrógeno. El valor del pH de determinada sustancia está directamente relacionado a la proporción de las concentraciones de los iones de hidrógeno [H+] e hidroxilo [OH-]. 
Si la concentración de H+ es mayor que la de OH-, el material es ácido; el valor del pH es menor que 7. 
Si la concentración de OH- es mayor que la de H+, el material es básico, con un pH con valor mayor que 7. 
Si las cantidades de H+ y de OH- son las mismas, el material es neutral y su pH es 7. Ácidos y bases tienen, respectivamente, iones de hidrógeno y de hidroxilo libres. La relación entre los iones de hidrógeno y de hidroxilo en determinada solución es constante para un dado conjunto de condiciones y cada uno puede ser determinado desde que se conozca el valor del otro.
Una lectura más precisa significa que es necesario utilizar un medidor de pH. Cuando está eligiendo un tester o medidor de pH, existen múltiples consideraciones relacionadas tanto al electrodo como al dispositivo que deben tenerse en cuenta. Asegúrese de encontrar un medidor de pH y un electrodo que mejor se ajuste a su área de trabajo.


El medidor de pH es un instrumento utilizado para medir la acidez o la alcalinidad de una solución, también llamado de pH. El pH es la unidad de medida que describe el grado de acidez o alcalinidad y es medido en una escala que va de 0 a 14, con 0 siendo una solución de ácido fuerte y 14 una solución básica fuerte.

Las informaciones cuantitativas dadas por el valor del pH expresan el grado de acidez de un ácido o de una base en términos de la actividad de los iones de hidrógeno. El valor del pH de determinada sustancia está directamente relacionado a la proporción de las concentraciones de los iones de hidrógeno [H+] e hidroxilo [OH-]. Si la concentración de H+ es mayor que la de OH-, el material es ácido; el valor del pH es menor que 7. Si la concentración de OH- es mayor que la de H+, el material es básico, con un pH con valor mayor que 7. Si las cantidades de H+ y de OH- son las mismas, el material es neutral y su pH es 7. Ácidos y bases tienen, respectivamente, iones de hidrógeno y de hidroxilo libres. La relación entre los iones de hidrógeno y de hidroxilo en determinada solución es constante para un dado conjunto de condiciones y cada uno puede ser determinado desde que se conozca el valor del otro.




\subsubsection{Sensor de oxigeno disuelto (DO)}

Toda la vida acu\'atica depende de la disponibilidad de oxígeno disuelto (DO) en el agua. Mientras que los organismos terrestres viven en una atm\'osfera compuesta aproximadamente de un 20\% de ox\'igeno, los organismos acu\'aticos sobreviven con una cantidad de ox\'igeno considerablemente menor. La solubilidad del ox\'igeno en agua dulce var\'ia entre 14.6 mg/L a 0$^{\circ}$C hasta aproximadamente 7 mg/L a 35$^{\circ}$C bajo una presi\'on de 760 mmHg. La concentraci\'on de ox\'igeno disuelto en agua est\'a determinada por la ley de Henry, que describe la relaci\'on de equilibrio entre la presi\'on parcial de ox\'igeno atmosf\'erico y la concentraci\'on de oxígeno en agua. Otros factores que influyen la concentraci\'on de ox\'igeno disuelto en agua son: la presi\'on atmosf\'erica (y por lo tanto la altitud sobre el nivel del mar), el contenido de sales en el agua, y la temperatura del agua. El contenido de ox\'igeno disuelto en cuerpos de agua puede disminuir significativamente por efecto de la respirac\'on, especialmente la microbiana, resultante de la degradaci\'on de compuestos org\'anicos. 

Traducido eso a un sensor cual permita detectar estos factores, se tiene la sonda galv\'anica de ox\'igeno disuelto que consta de una membrana de politetrafluoroetileno (PTFE), un \'anodo ba\~nado en un electrolito y un c\'atodo. Las mol\'eculas de ox\'igeno se desactivan a trav\'es de la membrana de la sonda a una velocidad constante (sin la membrana, la reacción es r\'apida). Una vez el las mol\'eculas de ox\'igeno han atravesado la membrana se reducen en el c\'atodo y se produce una peque\~na tensi\'on. Si no hay mol\'eculas de ox\'igeno presentes, la sonda emitir\'a 0 mV. A medida que aumenta el ox\'igeno, tambi\'en lo hace la salida de mV de la sonda. La sonda emite un voltaje diferente en presencia de ox\'igeno. Lo \'unico que es constante es que 0mV = 0 ox\'igeno []

emitite
\subsubsection{Sensor de conductividad eléctrica (CE) }
La salinidad de un suelo o agua, se refiere a la cantidad de sales presentes en solución, y puede ser estimada indirectamente mediante la medición de la conductividad eléctrica (CE). El valor de CE es influenciado por la concentración y composición de las sales disueltas. A mayor valor de CE, mayor es la salinidad presente. Es importante considerar que todos los fertilizantes inorgánicos son sales y por lo mismo tienen un efecto directo sobre la CE.

La salinidad es un fenómeno indeseable ya que afecta el crecimiento de las plantas  de varias maneras y por lo mismo, un aumento en la CE traerá como consecuencia una disminución de rendimiento.


\subsubsection{Sensor de potencial de oxido reduc\'on (OPR)}
Los sensores de potencial de reducción de la oxidación (ORP) miden la habilidad de una solución de actuar como agente oxidante o reductor. Para conseguir resultados exactos, es importante contar con una combinación correcta de sistema de referencia, unión y forma. METTLER TOLEDO ofrece sensores de ORP (Redox) con superficies de metal lisas y uniones de referencias únicas para garantizar unas mediciones fiables, incluso con muestras sucias.



\subsubsection{Sensor de temperatura (T)}
La temperatura se puede medir utilizando un sensor de temperatura de los diferentes tipos que existen. Todos ellos infieren la temperatura al detectar algún cambio en una característica física. Hay seis tipos de sensor de temperatura con los cuales es probable que el ingeniero se encuentre: termopares, dispositivos de temperatura resistivos (RTD y termistores), radiadores infrarrojos, dispositivos bimetálicos, dispositivos de dilatación de l\'iquido, y dispositivos de cambio de estado.
Un termopar es un sensor para medir la temperatura. Se compone de dos metales diferentes, unidos en un extremo. Cuando la unión de los dos metales se calienta o enfría, se produce una tensión que es proporcional a la temperatura. Las aleaciones de termopar están comúnmente disponibles como alambre.
Los termopares est\'an disponibles en diferentes combinaciones de metales o calibraciones para adaptarse a diferentes aplicaciones. Los tres más comunes son las calibraciones tipo J, K y T, de los cuales el termopar tipo K es el más popular debido a su amplio rango de temperaturas y bajo costo.


\section{Muestreo de Aguas.}
%[RR] información de Regina  



%% Revolución 222 tipos de aguas
%%
%%Coerencia de resiltado
%% Fiabilidad 
%% Estándar, regulador de transporte 
%%