%% Materiales, imprsion 3D

\chapter[Marco Te\'orico.]{Marco Te\'orico\label{etiqueta}}
%-------------------------------------------------------------------------------------
% [RR] En este cap quiero iniciar con empezar hablando un poco de lo que es calidad del agua, hablar sobre los factores físico - químicos quede deben de tenerse en cuenta, profundizar los parámetros físicos químicos que yo utilizo (temp, ph, conductividad eléctrica, opr, salinidad, total solidos disueltos, oxigeno disuelto.  .
% [RR]Luego quiero profundizar en cada sección:
% [RR]-> un poco sobre cada uno de los sensores, tipo, funcionamiento, 
% [RR]-> Tipos de muestreos manuales y automáticos, opciones comerciales, (énfasis al valor agregado de medicion a multinivel)


\section{Monitoreo de Agua.}
El agua, un recurso presente en la naturaleza, indispensable para la vida misma e infiere en la mayoría de los procesos industriales \cite{estudio_agua}, en el planeta el 2.5 \% del agua es dulce, de los cuales el 69 \% se encuentra en estado sólido en los polos y cumbres de la monta\~nas altas del alrededor del mundo. Tal como indican en \cite{junta-municipal-de-agua-potable-y-alcantarillado-de-mazatlan-no-date}, el 30 \% del agua dulce del mundial, se encuentra en la humedad del suelo y en los acuíferos profundos, el 1 \% del agua dulce en el mundo se encuentran por las cuencas hidrográficas en forma de arroyos y ríos y se depositan en lagos, lagunas y en otros cuerpos superficiales de agua y en acuíferos  \cite{junta-municipal-de-agua-potable-y-alcantarillado-de-mazatlan-no-date}.

Un alto porcentaje de los recursos h\'idricos alrededor del mundo están contaminados por diversas causas, haciendo que estos los ecosistemas el cual conforman se dañen gravemente y en el caso de no tomar las medidas de mitigación a tiempo, es posible que los da\~nos se tornen irreversibles, alterando gravemente al medio ambiente que a su vez un sin fin de problemas relacionados como, perdida de fauna, flora y paisaje, enfermedades a las poblaciones aledañas además de todo el impacto económico y social que conlleva.

Si bien los niveles de contaminación en el Paraguay son mínimos, no se debe dejar de lado el problema que se viene generando por la mala utilización de las tecnologías en la perforación de pozos en algunas zonas, y el que los graves problemas de contaminación en lugares lejanos de Brasil y Argentina, debido a la dinámica del agua, afectan la totalidad de la región.
\cite{saladuenas015}

\begin{table}[]
\begin{tabular}{lll}
\hline
\multicolumn{1}{c}{\textbf{Actividad}}                                     & \multicolumn{1}{c}{\textbf{Efectos}}                                                                 & \multicolumn{1}{c}{\textbf{Área}}                                                                        \\ \hline
Urbana e industrial                                                        & \begin{tabular}[c]{@{}l@{}}Contaminaci\'on superficial y \\ \\ acu\'iferos.\end{tabular}                 & \begin{tabular}[c]{@{}l@{}}REMA (Regi\'on Metropolitana \\ de Asunci\'on)\\ Cuenca de Ypacarai.\end{tabular} \\
Mataderos y frigor\'ificos                                                   & Aguas superficiales                                                                                  & REMA                                                                                                     \\
\begin{tabular}[c]{@{}l@{}}Expansi\'on Urbana \\ \\ desordenada\end{tabular} & Destrucci\'on de fuentes de agua                                                                       & \begin{tabular}[c]{@{}l@{}}Cuenca de Ypacarai\\ Cuenca del Ypoa\end{tabular}                             \\
\begin{tabular}[c]{@{}l@{}}Expansi\'on Urbana \\ \\ desordenada\end{tabular} & Aguas superficiales                                                                                  & REMA                                                                                                     \\
\begin{tabular}[c]{@{}l@{}}Cementera\\ Metal\'urgica\end{tabular}            & \begin{tabular}[c]{@{}l@{}}Contaminaci\'on superficial y \\ \\ acu\'iferos\end{tabular}                  & \begin{tabular}[c]{@{}l@{}}REMA\\ Villa Hayes\end{tabular}                                               \\
Prácticas agrícolas .                                                      & \begin{tabular}[c]{@{}l@{}}Colmataci\'on por erosi\'on y \\ \\ contaminaci\'on por pesticidas\end{tabular} & Todos los Dptos                                                                                          \\ \hline
\end{tabular}
\hfill \textbf{Fuente:} Extra\'ido de \cite{salas-duenas-2015}.
\end{table}

En el informe \cite{FAO_2015} publicado por la Organizaci\'on de las Naciones Unidas para la Alimentación y la Agricultura (FAO) indica, \textit{que existen registros y monitoreos de la calidad de las aguas, sea de forma privada, como por intervenciones del estado, (pedidos de la SEAM, controles de DIGESA). No obstante, esta información (recopilada por la SEAM) no es sistematizada, ni estudiada como para hacer un diagnóstico de la calidad de las aguas a nivel nacional}.

Conocer el estado real de los recursos h\'idricos se puede tornar una tarea bastante complicada, donde una cantidad de factores que pueden influir, desde la t\'ecnica del sensado o monitoreo, la ubucaci\'on de las sondas fijas en caso de monitoreo autom\'atico o en la recolecci\'on de muestras para an\'alisis posterior en el caso de muestreos manuales.

En el caso monitoreo manual, a efecto de este trabajo se considera que los trabajos manuales se realiza mediante la recolecc\'on de muestras biol\'ogica in situ, almacenamiento temporal y traslado a un laboratorio donde se realizan las lecturas mediante robustas y normalizadas metodolog\'ias de an\'alsis con sensores de alta calidad, fidelidad y certeza de calibraci\'on, el monitoreo manual involucra un importante despliegue logístico, técnicos, especialistas, recipientes de recolecci\'on, entre otros factores que pueden elevar bastante los costos de las campa\~nas de monitoreo, sin considerar las limitaciones que se podr\'ian presentar con respecto a la inaccesibilidad geogr\'afica para llegar al lugar de muestreo y las relativamente pocas muestras que se podr\'ian recolectar por motivos de almacenamiento y peso del material biol\'ogico.   

En otro sentido, Las sondas fijas utilizadas en el monitoreo no requieren de constantes despliegues logísticos, posteriores a su instalación, pero si requiere de una inversi\'on mucho más considerable y periódicos mantenimientos. 
Por su alto costo de instalación no es posible cubrir en su totalidad del recurso hídrico a ser analizado. % Consultar al PTI costo para análisis económico

El Paraguay es un pa\'is muy rico hidr\'ogicamente, por la cantidad de agua dulce en su territorio, la hidrología de Paraguay está formada por cascadas, ríos, lagos,  humedales. También cuenta con una importante reserva subterránea del acuífero guaraní, que se comparte con los países vecinos.

\textbf{Los Humedales del  Paraguay.}

Paraguay posee  una extensa \'area de ambientes h\'umedos, conocidos como pantanales, selvas de ribera, embalsados, esteros y otras tantas denominaciones que tienen en com\'un ser considerados humedales por la convenci\'on Ramsar\cite{salas-duenas-2015},
que entr\'o en vigor en Paraguay el 7 de octubre de 1995 y tiene actualmente 6 sitios designados como Humedales de Importancia Internacional, con una superficie de 785,970 hectáreas.\cite{ramsarWEB}

Las \'areas de humedales ocupan aproximadamente un \'area que equivale entre un $30\%$ a $40\%$ \cite{mereles1998humedales}, lo que representa un \'area que puede variar entre los 100.000 $Km^2$ y los 150.000 $Km^2$, respectivamente.

Algunos de los humedales del Paraguay, hacen parte de El Pantanal, que es considerado el mayor humedal del mundo, con unos 140,000 $Km^2$ y cubre \'areas en Brasil, Bolivia, y Paraguay. Forma las nacientes del R\'io Paraguay, considerado uno de los m\'as importantes reservorios de agua para los pa\'ises de la regi\'on. Es el mayor reservorio de plantas acu\'aticas del hemisferio occidental y contiene grandes concentraciones de peces y aves, lo que posiblemente le confiere el contener la mayor concentraci\'on de especies acu\'aticas del mundo \cite{willink2000biological}.

Como indica \cite{vera2000iniciativas}, estos humedales se encuentran distribuidos en las dos regiones naturales del pa\'is. La Occidental o Chaco, donde existen extensas \'areas con suelos formados b\'asicamente por arcillas y limos, los cuales no permiten la filtraci\'on del agua, facilitando la formaci\'on de extensos pantanales.

En su trabajo sobre los humedales \cite{fernandez2005humedales} considera que los humedales del chaco paraguayo son producto de la interacci\'on de las aguas de desborde del R\'io Paraguay al este, las inundaciones provenientes del r\'io Pilcomayo al sureste y las lluvias ocurridas al este de la regi\'on.

\textbf{Agua subterránea}

Paraguay cuenta con varios acu\'iferos, en la figura \ref{acuiferosPy} se pueden observar su distribuci\'on en el territorio Paraguayo,  considerando su nivel de explitaci\'on y el volumen de agua con lo que cuenta, los m\'as importantes son \cite{FAO_2015}:
\begin{itemize}
\item Acu\'ifero Yrendá, se estima que el Pilcomayo contribuye con 860 millones de $m^{3}$, y la recarga total se estima en  2.46$km^{3}$.
\item Acu\'ifero Pati\~no: ubicada en la zona central del pais, según el Balance Hídrico Integrado de 2005, al Pati\~no ingresan 175.8 millones de $m^{3}$ y se extraen 249 millones de $m^{3}$. 
\item El Acu\'ifero Guaran\'i es compartido con Brasil, Argentina y Uruguay, tiene un área total de 1,150,000 $km^{2}$, con 150,000 $km^{2}$ de recarga natural, arrojando un volumen promedio de 160 $km^{3}/a\~no$ en total. Corresponde a Paraguay 67,000 $km^{2}$, con 37.000 $km^{2}$ de recarga, y un volumen de 39 $km^{3}/a\~no$. 
\end{itemize}
Se estima que el volumen de los tres acu\'iferos m\'as importantes del pa\'is rondar\'ian los 41,64 $km^{3}$. \cite{alvarez-2014}

El sistema del Acuífero Guaraní, tienen una superficie total estimada de aproximadamente 1,2 millones de $km^{2}$, de los cuales unos 71.700 $km^{2}$ corresponderían a territorio paraguayo y se estima que en total unas 15 millones de personas viven en su área de influencia.\cite{salas-duenas-2015}

Los principales usuarios del agua en el Paraguay son el consumo dom\'estico de la poblac\'on, la ganader\'ia, la agricultura con riego y la industria. Entre los usos no consuntivos se tienen las represas hidroel\'ectricas y la navegaci\'on que depende de los niveles del r\'io.\cite{groundwater_caracteristicas_2021}
Hay una prevalencia del suministro de agua potable por medio de agua subterr\'anea, el 80$\%$ del abastecimiento de comunidades en el interior del pa\'is es con agua subterr\'anea. Esto genera una fuerte presi\'on sobre los acu\'iferos, con el consecuente peligro de contaminaci\'on que estos pozos representan, (en ocasiones construidos por el mismo Estado, sin cumplir los requerimientos t\'ecnicos y legales). El caso m\'as cr\'itico es el acu\'ifero Pati\~no, ubicado en la zona del departamento
Central con la mayor densidad demogr\'afica.\cite{alvarez-2014}

    \begin{figure}[H]
        \centering
        \includegraphics[scale=1]{Imagenes/cap2/images.jpg}
        \caption {Acuíferos del Paraguay} 
        \textbf{Fuente:}
        \cite{alvarez-2014} 
        \label{acuiferosPy}
    \end{figure}




\section{Normativas paraguayas}

La Secretar\'ia del Ambiente (SEAM), a trav\'es de su Direcci\'on General de Protecci\'on y Conservaci\'on de los Recursos H\'idricos (DGPCRH), protege los humedales y gestiona el manejo de los mismos con el fin de su conservaci\'on. Asimismo, la Direcci\'on General de Protecci\'on y Conservaci\'on de la Biodiversidad (DGPCB) es Punto Focal y autoridad de la Ley Nº 350/94 "Que aprueba la Convenci\'on Relativa a los Humedales de Importancia Internacional, especialmente como h\'abitat de aves acu\'aticas”, seg\'un lo establece la Ley Nº 1561/00 que crea el Sistema Nacional del Ambiente,el Consejo Nacional del Ambiente y la Secretar\'ia del Ambiente, en su art/'iculo 14, inciso J.\cite{direccion_general_de_proteccion_y_sitios_nodate}

El Decreto Reglamentario 222/02 establece valores límites para determinados parámetros y representa la referencia de calidad del agua a nivel nacional. Este decreto clasifica el estado de los cuerpos de agua continentales en 4 clases:
\begin{itemize}
    \item Clase 1:
    Aguas destinadas o que puedan ser destinadas al abastecimiento de agua potable a poblaciones con tratamiento convencional.
    \item Clase 2:
    \begin{itemize}
        \item Aguas destinadas al riego de hortalizas o plantas frutícolas u otros cultivos destinados al consumo humano en su forma natural, cuando éstas son usadas a través de sistemas de riego que provocan el mojado del producto.
        \item Aguas destinadas a recreaci\'on por contacto directo con el cuerpo humano.
    \end{itemize}
    \item Clase 3:
    Aguas destinadas a la preservaci\'on de los peces en general y de otros integrantes de la flora y fauna h\'idrica, o tambi\'en aguas destinadas al riego de cultivos cuyo producto no se consume en forma natural o en aquellos casos que siendo consumidos en forma natural se apliquen sistemas de riego que no provocan el mojado del producto.
    \item Clase 4
    Aguas correspondientes a los cursos o tramos de cursos que atraviesan zonas urbanas o suburbanas que deban mantener una armonía con el medio, o también aguas destinadas al riego de cultivos cuyos productos no son destinados al consumo humano en ninguna forma.

\end{itemize}
%%% Tablas 

\begin{table}[htpb]
\caption{L\'imites de par\'ametros f\'isico qu\'imico de aguas. Seg\'un Resoluci\'on 222/02 - SEAM}
\label{tab:Limites222}

\begin{tabular}{lcccc}
\toprule
Par\'ametros                         & Clase I & Clase II & Clase III & Clase IV \\ \noalign{\hrule height 2pt}
                                    &           &            &            &            \\
Oxígeno Disuelto ($mg$ $O_{2}.L^{-1}$) & $\geq$ 6 & $\geq$ 5   & $\geq $4    & $\geq$ 2  \\
                                    &           &            &            &            \\
pH (Unidad de $pH$)                   & 6-9       & 6-9        & 6-9        & 6-9        \\
                                    &           &            &            &            \\
Turbidez ($NTU$)                      & $\leq$ 40 & $\leq$ 100 & $\leq$ 100 & $\leq$ 100 \\
                                    &           &            &            &            \\
Color (real) ($mg$ $Pt.L^{-1}$)       & 15        & 75         & 75         & 75         \\
                                    &           &            &            &            \\
Dureza Total ($mg$ $CaCO_{3}.L^{-1}$) & 750       & 750        & 750        & 750        \\
                                         &           &            &            &            \\
DBO-5 (20º C) ($ mg$ $O_{2}.L^{-1}$)      & 3,0       & 5,0        & 10         &            \\
                                        &           &            &            &            \\
Nitr\'ogeno Total en agua ($mg.L^{-1}$) & 0,3       & 0,6        & 0,6        & 0,6        \\
                                        &           &            &            &            \\
F\'osforo Total en agua ($mg.L^{-1}$)   & 0,025     & 0,05       & 0,05       & 0,05       \\
                                        &           &            &            &            \\
Nitrógeno Amoniacal ($mg.L^{-1}$)       & 0,0165    & 0,0165     & 0,0165     & 0,0165     \\
                                        &           &            &            &            \\
Nitrógeno de Nitritos ($mg/L$)          & 1         & 1          & 1          & 1          \\
                                        &           &            &            &            \\
Nitrógeno de Nitratos ($mg/L$)          & 10        & 10         & 10         & 10         \\
                                    &           &            &            &            \\ 
Cloruro ($mg /L$)                       & 250       & 250        & 250        & 250        \\
                                        &           &            &            &            \\
Sodio ($mg/L$)                          & 200       & 200        & 200        & 200        \\
                                        &           &            &            &            \\
Hierro Ferroso ($mg/L$)                 & 0,3       & 0,3        & 0,3        & 0,3        \\
                                        &           &            &            &            \\
Sulfatos ($mg/L$)                       & 250       & 250        & 250        & 250       \\
\bottomrule
\end{tabular}
\\
\bigskip
\small \textit{Nota}. Extra\'ido de los Art\'iculos 2$^{o}$,3$^{o}$,4$^{o}$,5$^{o}$.\textit{Fuente}. \cite{secretaria-del-ambiente-2022}.
\end{table}

Monitoreo del agua potable: los parámetros químicos comunes incluyen pH, nitratos y oxígeno disuelto. 
La medición de O2 (o DO) es un indicador importante de la calidad del agua. 
Los cambios en los niveles de oxígeno disuelto indican la presencia de microorganismos de aguas residuales, escorrentías urbanas o agrícolas o descargas de fábricas. 
Un nivel adecuado de ORP minimiza la presencia de microorganismos como  E. coli, Salmonella, Listeria . 
Los niveles de turbidez por debajo de 1 NTU indican la pureza adecuada del agua potable.
Detección de fugas químicas en ríos : pH extremo o valores bajos de OD indican derrames químicos debido a problemas en la planta de tratamiento de aguas residuales o en la tubería de suministro.
Medición remota de piscinas : la medición del potencial de oxidación-reducción (ORP), el pH y los niveles de cloro del agua puede determinar si la calidad del agua en piscinas y spas es suficiente para fines recreativos.
Niveles de contaminación en el mar : la medición de los niveles de temperatura, salinidad, pH, oxígeno y nitratos proporciona información para los sistemas de detección de calidad en el agua de mar.
Prevención de la corrosión y depósitos de cal:  Controlando la dureza del agua podemos evitar la corrosión y depósitos de cal en lavavajillas y dispositivos de tratamiento de agua como calentadores. 
La dureza del agua depende de: pH, temperatura, conductividad y concentraciones de calcio (Ca + ) / magnesio (Mg 2+ ).
Cultivo de abetos / Monitoreo de tanques de peces / Criadero / Acuicultura / Acuaponía: Medición de las condiciones del agua de animales acuáticos como caracoles, peces, cangrejos de río, camarones o langostinos en tanques. 
Los valores importantes son el pH, el oxígeno disuelto (DO), el amoníaco (NH 4 ), el nitrato (NO 3 - ), el nitrito (NO 2 - ) y la temperatura del agua.
Hidroponía : las plantas que toman los nutrientes directamente del agua necesitan un pH preciso y niveles de oxígeno en el agua (OD) para obtener el máximo crecimiento.

Las sondas del sensor miden más de 12 parámetros químicos y físicos de la calidad del agua, como pH, nitratos (NO3), iones disueltos (fluoruro (F - ), calcio (Ca 2+ ), nitrato (NO 3 - ), cloruro (Cl - ), Yoduro (I - ), Cúprico (Cu 2+ ), Bromuro (Br - ), Plata (Ag + ), Fluoroborato (BF 4 - ), Amoníaco (NH 4 ), Litio (Li + ), Magnesio (Mg 2+ ) , Nitrito (NO 2- ), Perclorato (ClO 4 ), Potasio (K + ), Sodio (Na +) oxígeno disuelto (OD), conductividad (salinidad), potencial de oxidación-reducción (ORP), turbidez, temperatura, etc. Los contaminantes pueden detectarse y tratarse en tiempo real para garantizar una buena calidad del agua en toda la red de suministro de agua. Los valores extremos de pH pueden indicar derrames de productos químicos, problemas en la planta de tratamiento o problemas en las tuberías de suministro. Los niveles bajos de OD pueden indicar la presencia de microorganismos debido a escorrentías urbanas / agrícolas o derrames de aguas residuales. El ORP mide qué tan bien está funcionando la desinfección del agua.

\section{Calidad del agua.}
%Tema pendiente de ver 
%Preguntar en http://habitat.aq.upm.es/dubai/00/bp561.html sobre los parametros y estandares Py.
%%%%%%%%%

El término calidad del agua se utiliza para describir el estado del agua, incluida su características químicas, físicas y biológicas, generalmente con respecto a su idoneidad para un propósito particular (es decir, beber, nadar o pescar)~\cite{waterquality}.

La calidad del agua se puede considerar como una medida de la idoneidad del agua para un uso particular en función de determinadas características físicas, químicas y biológicas. 
Para determinar la calidad del agua, los técnicos primero miden y analizan las características del agua, como la temperatura, el contenido de minerales disueltos y la cantidad de bacterias. 
Luego, las características seleccionadas se comparan con normas y pautas numéricas para decidir si el agua es adecuada para un uso particular. 
La calidad es el estado de las características físicas, biológicas y químicas del agua basados en condiciones establecidas

Los parámetros óptimos de las características físicas, químicas y biologías del agua, se establecen según las normas de los organismos reguladores. 
En Paraguay el organismo regulador 
% [RR] Concluir esta sección con los parámetros de algún organismo 

\section{Sensores para calidad de agua.}

Las características químicas, físicas y biológicas del agua se combinan para formar lo que denominaremos calidad del agua. 
Mínimos cambios en estas características pueden poner en peligro los ecosistemas. 
Para preservar su calidad, la monitorización precisa de los parámetros del agua como la conductividad, el pH, la salinidad, la temperatura, el oxígeno disuelto (OD), OPR, son fundamentales. 
A continuaci\'on  describiremos los sensores de calidad del agua que son adecuados tanto para puntos simples requisitos de muestreo y proyectos complejo.

Existen varios parámetros que pueden estudiarse para indicar la calidad del agua.
Estos parámetros pueden medirse ya sea de características físicas como el pH, conductividad, o temperatura; niveles de varios nutrientes en agua, como nitratos y fosfatos; o de algunos elementos y compuestos del agua, como el oxígeno disuelto. 

Existen una gran variedad de sensores para la obtención de las mediciones de los índices de calidad del agua, en el presente TFG abordaremos los sensores de pH, conductividad eléctrica, oxigeno disuelto, potencial de oxido reducción y temperatura. 

\subsection{Medición de pH}
\subsubsection{¿Qué es pH?}

El valor de pH describe la actividad de los iones de hidrógeno en soluciones. 
Por análisis químicos se sabe que el pH siempre se encuentra en una escala de 0 a 14. 
Es importante decir que el pH mide el grado de acidez o de alcalinidad pero no determina el valor de la acidez ni de la alcalinidad \cite{sierra_ramirez_calidad_2011}. 
Con base en esta escala de pH, los líquidos se caracterizan por ser ácidos, alcalinos o neutros; una solución que no es ácida ni alcalino es neutro, lo que corresponde a un valor de 7 en la escala de pH. 
La acidez indica una mayor actividad de los iones de hidrógeno y un valor de medición de pH inferior a 7. 
Las soluciones alcalinas se caracterizan por un ion de hidrógeno más bajo actividad o mayor actividad de iones de hidróxido, respectivamente, y una medición de pH valor por encima de 7.
La escala de pH es logarítmica. Una diferencia de una unidad de medida de pH representa un aumento o reducción de 10 veces, o 10 veces, de la actividad de los iones de hidrógeno en la solución~\cite{covington_definition_1985}.

\begin{equation} \label{ecuacionpH} 
%%\[pH= -\log_{10}[H^{+}\]
pH=-log_{10}[H^{+}]
\end{equation}


\subsubsection{¿Como medirlo?}
Existen varias formas de medir esta magnitud química, se puede medir utilizando sistemas de medición electroquímicos, papel tornasol o indicadores, electrodos, amperímetros, ISFET entre otros. 

\begin{itemize}
    \item Papel tornasol: La forma más sencilla de medir el pH es usar los indicadores o tiras de tornasol, son simples y económicos. 
    Desafortunadamente, en muchos casos el papel tornasol y indicadores no son lo suficientemente precisos para realizar mediciones de pH de alta calidad. 
    Ambos métodos proporcionan un pH basado en una reacción química que hace que cambie el color. 
    Si la muestra de papel o líquido cambia de color, se debe comparar con la escala de colores y allí, según el usuario, elegir el que mejor se adapte.
    
    \begin{figure}[H]
        \centering
        \includegraphics[width=100mm, height=65mm]{Imagenes/cap2/Papel-tornasol-min.png}
        \caption {Papel tornasol o papel pH. \textbf{Fuente:} \cite{noauthor_papel-tornasol-min-768x432png_nodate} }
        \label{fig:tornasol}
    \end{figure}
    
    \item Amperometría: La diferencia en la concentración de iones de hidrógeno (fuera de la sonda vs. dentro de la sonda) crea una corriente muy pequeña. 
    Esta corriente es proporcional a la concentraci\'on de iones de hidrógeno en el líquido medido \cite{Atlas_pH}. 
    La ventaja de la amperometría como método de medición del pH es que es fácil de usar. 
    En mediciones amperométricas de pH generación de hidrógeno ocurre en un metal noble, cuando se combina con un metal menos noble, se forma una c\'elula galvánica de distribución de energía. 
    Debido a que se generan iones de hidrógeno, la corriente de la c\'elula depende del valor del pH. 
    Las desventajas de este método son que las diferencias en la composición de la muestra crean errores muy grandes en las mediciones de pH y el método no puede ofrecer resultados fiables en ácidos y bases extremadamente concentrados, debido a efectos relacionados con la membrana de cristal. 
    
    \begin{figure}[H]
        \centering
        \includegraphics[width=100mm, height=65mm]{Imagenes/cap2/amperimetrico.png}
        \caption {Sensor OPTISENS CL 1100.} \textbf{Fuente:}
        \cite{ampe_sensores_nodate} 
        \label{fig:amperimetrica}
    \end{figure}
    
    \item ISFET: \textit{Ion Sensitive Field Effect Transitor} o transistor de efecto campo sensible a iones en un sensor electroqu\'imico que reaccionan a cambios en la actividad de un ion dado, un m\'etodo relativamente nuevo para la medici\'on del valor de pH \cite{duroux_ion_1991}.
    
    Es un transistor con fuente de poder y desagüe, dividido por un aislador. 
    Este aislador (puerta) está hecho de un óxido metálico donde los iones de hidrógeno se acumulan de la misma manera que un electrodo. 
    La carga positiva que se acumula fuera de la puerta se ``refleja'' en el interior la puerta por una carga negativa igual generada. 
    Una vez que esto sucede, la puerta comienza a conducir electricidad. 
    Cuanto menor sea el valor de pH, más iones de hidrógeno se acumulan y más corriente puede fluir entre la fuente y el drenaje. 
    El ISFET Los sensores, similares a los electrodos de pH de vidrio, actúan de acuerdo con la ecuación de Nernst.
    La ventaja de un ISFET es que es muy pequeño. 
    El transistor de efecto de campo real (FET) es de solo \(0.2 mm^{^{2}}\). 
    La desventaja de usar un ISFET para mediciones de pH es que tienen una durabilidad comparativamente corta y una baja duración a largo plazo~\cite{li_chapter_2019}.
    
    \begin{figure}[H]
        \centering
        \includegraphics[width=70mm, height=45mm]{Imagenes/cap2/ISFET_Euro.jpg}
        \caption {Sensor de pH-ISFET MSFET 3330. \textbf{Fuente:}
        \cite{microsens_nodate} }
        \label{fig:isfet}
    \end{figure}
    
    \item Electrodos: El método más común de medición del valor de pH es el uso de electrodos.
    Estos dispositivos de medición de pH son sensores electroquímicos que constan de un electrodo de medición y un electrodo de referencia. 
    El electrodo de medición de pH está hecho de vidrio especial que, debido a sus propiedades superficiales, es particularmente sensible a iones de hidrógeno. 
    El electrodo de medición de pH se llena con una solución tampón de un valor de pH de 7. 
    Al colocar el electrodo de medición de pH en una prueba soluci\'on, el cambio de voltaje se mide con el electrodo de pH comparando el voltaje medido al electrodo de referencia estable. 
    Este cambio se registra por el medidor de pH y se convierte en el valor de medición de pH que se muestra.
    
    \begin{figure}[H]
        \centering
        \includegraphics[width=60mm, height=45mm]{Imagenes/cap2/ph700.png}
        \caption {PH700 Benchtop pH Meter Kit. \textbf{Fuente:}
        \cite{ph700_nodate} }
        \label{fig:ph700}
        \end{figure}
\end{itemize}

    De todos estos métodos de medición del valor de pH,  el mejor es el uso de electrodos de pH. 
    No hay otro sistema de medición de pH que proporcione mejor confiabilidad, precisión y velocidad de la medición del pH en todo el pH distancia. 
    La desventaja mínima de usar electrodos de vidrio para pH para este método de medición de pH es el hecho de que los electrodos de vidrio son delicados y deben manipularse con cuidado~\cite{li_chapter_2019}.

\subsubsection{Sensor de potencial de hidrógeno (pH)}

El medidor de pH es un instrumento utilizado para medir la acidez o la alcalinidad de una solución, también llamado de pH. El pH es la unidad de medida que describe el grado de acidez o alcalinidad y es medido en una escala que va de 0 a 14, con 0 siendo una solución de ácido fuerte y 14 una solución básica fuerte.
Es una manera de evaluar que tan adecuada es el agua para una planta o animal.  Si el agua es demasiado ácida o básica, ya sea a por contaminantes naturales o de origen humano, puede ver un impacto profundamente negativo a la vida acuática.  Un pH normal en cuerpos de agua tiene un valor de entre 5.0 a 9.0, pero de manera ideal debería estar en un rango de entre 6.0 a 8.0. %poner refdel 2222

\subsubsection{¿Cómo medir el pH?}
Test de pH comunes cómo los test kit químicos y tiras de tornasol, son simples y económicos. 
Aun así, estos métodos tienen inconvenientes que pueden causar resultados imprecisos. 
Ambos métodos entregan un valor de pH basado en una reacción química que resulta en el cambio de color. 
Cuando su papel o muestra líquida cambia de color, debe compararlo con la guía de color y allí seleccionar la que más se ajuste según su criterio.
Las informaciones cuantitativas dadas por el valor del pH expresan el grado de acidez de un ácido o de una base en términos de la actividad de los iones de hidrógeno. 
El valor del pH de determinada sustancia está directamente relacionado a la proporción de las concentraciones de los iones de hidrógeno [H+] e hidroxilo [OH-]. 
Si la concentración de H+ es mayor que la de OH-, el material es ácido; el valor del pH es menor que 7. 
Si la concentración de OH- es mayor que la de H+, el material es básico, con un pH con valor mayor que 7. 
Si las cantidades de H+ y de OH- son las mismas, el material es neutral y su pH es 7. Ácidos y bases tienen, respectivamente, iones de hidrógeno y de hidroxilo libres. 
La relación entre los iones de hidrógeno y de hidroxilo en determinada solución es constante para un dado conjunto de condiciones y cada uno puede ser determinado desde que se conozca el valor del otro.
Una lectura más precisa significa que es necesario utilizar un medidor de pH. 
Cuando está eligiendo un tester o medidor de pH, existen múltiples consideraciones relacionadas tanto al electrodo como al dispositivo que deben tenerse en cuenta. 
Asegúrese de encontrar un medidor de pH y un electrodo que mejor se ajuste a su área de trabajo.


El medidor de pH es un instrumento utilizado para medir la acidez o la alcalinidad de una solución, también llamado de pH. El pH es la unidad de medida que describe el grado de acidez o alcalinidad y es medido en una escala que va de 0 a 14, con 0 siendo una solución de ácido fuerte y 14 una solución básica fuerte.

Las informaciones cuantitativas dadas por el valor del pH expresan el grado de acidez de un ácido o de una base en términos de la actividad de los iones de hidrógeno. El valor del pH de determinada sustancia está directamente relacionado a la proporción de las concentraciones de los iones de hidrógeno [H+] e hidroxilo [OH-]. Si la concentración de H+ es mayor que la de OH-, el material es ácido; el valor del pH es menor que 7. Si la concentración de OH- es mayor que la de H+, el material es básico, con un pH con valor mayor que 7. Si las cantidades de H+ y de OH- son las mismas, el material es neutral y su pH es 7. Ácidos y bases tienen, respectivamente, iones de hidrógeno y de hidroxilo libres. La relación entre los iones de hidrógeno y de hidroxilo en determinada solución es constante para un dado conjunto de condiciones y cada uno puede ser determinado desde que se conozca el valor del otro.


\subsection{Medición de potencial ORP-REDOX}
\subsubsection{¿Que es ORP?}
El potencial de oxidacion-reduccion (ORP) se define como la fuerza electromotriz entre un electrodo de metal noble y un electrodo de referencia cuando se sumergen en una solución.
Los electrodos de ORP son inertes y miden la relación entre las actividades de las especies oxidadas y las reducidas presentes~\cite{d19_committee_test_nodate}.
También conocido como redox, es una medida en milivoltios y mide el potencial de intercambio de electrones, si una sustancia (generalmente liquida) se oxida o se reduce. 
Los oxidantes siempre tendrán un valor de ORP positivo, mientras que los reductores siempre tendrán un valor de ORP negativo. 
El ORP se usa comúnmente para medir la limpieza en los sistemas de agua y la descomposición de productos de desecho, escombros y contaminantes. 
La composición química del agua subterránea y los sistemas de acuíferos contaminados se ven afectados por procesos de reducción de oxidación (redox)~\cite{wator_redox_2020}.

\subsubsection{¿Cómo medirlo?}
Un sensor de ORP consta de un electrodo de ORP y un electrodo de referencia.
Los electrodos de ORP miden de forma fiable el ORP en casi todas las soluciones acuosas y, en general, no están sujetos a la interferencia de la solución por el color, la turbidez, la materia coloidal y la materia en suspensión~\cite{d19_committee_test_nodate}.

El electrodo OPR: el principio detras de la medicion de ORP es el uso de un electrodo de metal inerte (platino, a veces oro), que, debido a su baja resistencia, cedera electrones a un oxidante o aceptara electrones de un reductor. 
El electrodo de ORP continuará aceptando o cediendo electrones hasta que desarrolle un potencial, debido a la carga acumulada, que es igual al ORP de la solución. 
La precisión típica de una medida de ORP es d\'e \( \pm5 mV\).

El electrodo de referencia:  ser el mismo electrodo de plata-cloruro de plata utilizado con mediciones de pH~\cite{li_chapter_2019}.

    \begin{figure}[H]
        \centering
        \includegraphics[width=80mm, height=80mm]{Imagenes/cap2/ORP_Sensor_Image.jpg}
        \caption {Diagrama del sensor OPR. \textbf{Fuente:}
        \cite{orp_sensor_measure_nodate} }
        \label{fig:opr}
    \end{figure}


\subsubsection{Sensor de potencial de \'oxido reducción (OPR)}
Los sensores de potencial de reducción de la oxidación (ORP) miden la habilidad de una solución de actuar como agente oxidante o reductor. 
Para conseguir resultados exactos, es importante contar con una combinación correcta de sistema de referencia, unión y forma. METTLER TOLEDO ofrece sensores de ORP (Redox) con superficies de metal lisas y uniones de referencias únicas para garantizar unas mediciones fiables, incluso con muestras sucias.

\subsection{Temperatura}
\subsubsection{¿Qué es temperatura?}
La temperatura es una de las medidas más comunes en la vida diaria. 
En el contexto de calidad de agua, la temperatura puede proveer un indicio de las condiciones de vida para plantas acuáticas y animales.  
Las temperaturas templadas se consideran generalmente benéficas para el crecimiento de la población acuática. 
De cualquier forma, después de cierto punto la temperatura puede tener un efecto contrario, contribuyendo a declinar la diversidad biológica en cuerpos de agua.

\subsubsection{¿Porque la medición de temperatura es importante?}
Organismos acuáticos cómo los peces y plancton son de agua fría, así que la temperatura en el agua tiene un impacto directo en la temperatura de su cuerpo. 
Estos organismos tienen rangos de temperatura en los cuales pueden sobrevivir y desarrollarse. 
A medida que la temperatura alcanza el límite superior o su rango para un organismo, la actividad biológica estará en su tope. 
La actividad disminuirá cuando se alcance el punto mínimo del rango. 
Si la temperatura excede el rango aceptable para el organismo, la cantidad disponible de oxígeno puede ser demasiado baja para sustentarlos. 
Esto se debe a que el agua caliente tiene un punto de saturación de ox\'igeno menor al agua fría. 
Si la temperatura está por debajo del rango aceptable, no hay suficiente actividad para el crecimiento de los organismos. 
La temperatura alta también contribuye al crecimiento y floración de algas. 
El oxígeno se consume en la medida que bacterias descompongan estos brotes, lo que reduce la cantidad de oxígeno disuelto disponible.
La temperatura en varios cuerpos de agua se basa en la hora del día y la cantidad de luz solar calentando la superficie. 
Las temperaturas aceptables también varían dependiendo del tipo de río o corriente de agua que desee monitorear. 
Esto depende de la fuente de la cuenca que alimenta la corriente. 
Si se alimenta la corriente con agua de manantial, por ejemplo, la temperatura normal de la corriente puede ser adecuada (menor a 68 68$^\circ$ F). 
Una corriente se considera templada si tiene una temperatura promedio superior a los 68$^\circ$ F, pero menor a los 89$^\circ$ F. 
La temperatura también puede verse influenciada por el flujo y el cuerpo de agua. 
Si el flujo de agua se incrementa, quizá cómo resultado de una fuerte lluvia, se puede esperar que la temperatura disminuya. 
El incremento en la corriente tiene cómo efecto reducir la temperatura en el agua.

La polución por temperatura, también conocida cómo polución térmica, puede ser causada por vertimientos de agua calentada en asfalto o concreto. 
Esta también puede proceder de efluentes industriales que sean descargados en cuerpos de agua, o agua que sea usada cómo refrigerante en plantas de energía nuclear. 
Estos efluentes significan que el agua en el que son descargadas incrementará la temperatura general del cuerpo de agua.  
La temperatura también puede asociarse a la turbidez.  
Ya que la cantidad de luz absorbida incrementa a medida que el agua se oscurece, la temperatura también aumentará.

\subsubsection{¿Cómo medir la temperatura?}
Muchos termómetros simples usan un termistor.  
El termistor es un dispositivo semiconductor cuya resistencia varía en función de la temperatura.  
A medida que la temperatura incrementa, la resistencia disminuye.  
La resistencia medida por el termistor se convierte en un valor que se muestra ya sea en la escala Celsius o Fahrenheit.  
Los sensores termistores son adecuados para rangos de temperatura desde -50° a 150°C (-58° a 302°F).

\subsubsection{Calibración de temperatura}
Muchos medidores se calibran en fábrica para las lecturas de temperatura. 
Es una buena práctica revisar esta calibración al menos una vez al año en un laboratorio, y asegurar que el sensor de temperatura funciona de manera adecuada.

\subsubsection{Sensor de temperatura (T)}
La temperatura se puede medir utilizando un sensor de temperatura de los diferentes tipos que existen. 
Todos ellos infieren la temperatura al detectar algún cambio en una característica física. 
Hay seis tipos de sensor de temperatura con los cuales es probable que el ingeniero se encuentre: termopares, dispositivos de temperatura resistivos (RTD y termistores), radiadores infrarrojos, dispositivos bimetálicos, dispositivos de dilatación de l\'iquido, y dispositivos de cambio de estado.
Un termopar es un sensor para medir la temperatura. 
Se compone de dos metales diferentes, unidos en un extremo. Cuando la unión de los dos metales se calienta o enfría, se produce una tensión que es proporcional a la temperatura. 
Las aleaciones de termopar están comúnmente disponibles como alambre.
Los termopares están disponibles en diferentes combinaciones de metales o calibraciones para adaptarse a diferentes aplicaciones. 
Los tres más comunes son las calibraciones tipo J, K y T, de los cuales el termopar tipo K es el más popular debido a su amplio rango de temperaturas y bajo costo.

\subsection{Conductividad (CE) / Total de sólidos disueltos (TDS)}

\subsubsection{¿Qué es la conductividad?}
La conductividad eléctrica (CE) mide que tan bien una sustancia puede transmitir una corriente eléctrica. 
Pequeñas partículas cargadas, llamadas iones, pueden ayudar a transportar la corriente eléctrica a través de la substancia. 
Estos iones pueden estar cargados positiva, o negativamente. A mayor cantidad de iones disponibles, mayor será la conductividad; menores iones resultarán en una menor conductividad. 
La CE se reporta de manera habitual en milliSimens por centímetro (mS/cm).

El total de sólidos disueltos (TDS) es la cantidad de sustancias disueltas en soluciones. 
Las mediciones permiten conocer las sustancias orgánicas e inorgánicas disueltas en el líquido. 
Los resultados de esta lectura se muestran en miligramos por litro (mg/L), partes por mill\'on (ppm), gramos por litro (g/L), o partes por mil (ppt).

\subsubsection{¿Por qu\'e la medici\'on de conductividad es importante?}
La conductividad eléctrica (CE) es otra manera de evaluar la calidad del agua, ya que al incrementar la presencia de total de sólidos disueltos (TDS), expresada en la CE, puede ser un indicador de contaminantes. 
La CE puede verse afectada por los carbonatos presentes en el agua caliza, contaminantes humanos como aguas residuales u otro tipo de fuente cómo sistemas de pozos sépticos o residuos de agricultura.

Altas concentraciones de TDS pueden reducir la calidad de agua y causar problemas en el balance de agua para organismos individuales. 
Por otra parte, bajas concentraciones pueden limitar el crecimiento de vida acuática. 
Algunos de los efectos comentados para parámetros como el dióxido de carbono y acidez tienen relevancia para la CE, cómo su impacto negativo en la fotosíntesis. 
Esto se debe a que al incrementar la cantidad de sólidos el agua se oscurecerá, reduciendo la tasa de fotosíntesis. 
La CE proveerá un indicio del total de sólidos disueltos, del cual el total de sales disueltas es un componente. 
Si el nivel de sales en el TDS es alto, esto también puede contribuir a acidificar el agua. 
De cualquier forma, si el nivel de carbonatos en la lectura de TDS son altos, esto podría contribuir a incrementar la alcalinidad, lo que puede ayudar a proteger el agua ante cambios ácidos. 
Esta es una buena respuesta de interrelación, entre los parámetros de calidad de agua.

Niveles aceptables de CE en ríos y corrientes de agua varía dependiendo del tipo de sólidos disueltos presentes y esto determina el uso de la corriente, ya sea para pesca, nado o como una fuente de agua potable.


Corriente
Es importante comprender la relación entre TDS y sólidos totales. Los sólidos totales se refieren a toda la materia sólida, ya sea suspendida o disuelta en agua. 
Los sólidos disueltos no son visibles en el agua, ya que al disolverse se vuelven parte de la solución. 
Los TDS son una medida de las sustancias disueltas en agua que están en una muestra de agua. 
En una muestra recolectada en un rio, estas sustancias disueltas se conocen cómo solutos, y el agua es llamada solvente.

\subsubsection{¿Cómo medir la conductividad?}
La mejor manera de medir la conductividad es con el uso de un medidor. 
Dos electrodos que aplican un voltaje AC se ubican en la solución. 
Esto crea una corriente que depende de la conductividad natural de la solución. 
El medidor lee esta corriente y muestra la conductividad (CE) o ppm (TDS)

\subsubsection{Calibración de conductividad en campo}
Es importante calibrar la conductividad antes de analizar la muestra. 
Esto se debe a que recubrimientos aceitosos y contaminantes biológicos puede cambiar la geometría celular, resultando en una desviación en la constante celular. 
Antes de realizar la calibración de conductividad, siempre inspeccione el sensor de CE por residuos u obstrucciones.

La mayoría de los medidores se calibran con un solo estándar, procurando que este cerca de la conductividad de la muestra a medir. 
Un segundo estándar puede usarse para revisar la linealidad del instrumento en el rango de medición.

\subsubsection{Procedimiento de calibración}:
Llene el beaker con suficiente estándar para cubrir la unión del electrodo (cerca de 75 mL de un beaker de 100 mL) 
Vierta solución adicional en un segundo beaker para enjuagar el sensor.
Ubique el electrodo en el beaker de enjuague y asegúrese de que los canales del sensor de CE estén llenos con estándar fresco al enjuagar y agitar la sonda en el beaker unas cuantas veces.
Ubique la sonda en el beaker de calibración y golpee suavemente para liberar burbujas atrapadas.
Confirme el punto de calibración cuando la lectura sea estable, o cuando los dígitos no cambien por al menos 5 segundos. (Algunos medidores requerirán que ingrese el valor del estándar de conductividad)
La calibración esta completa. Enjuague la sonda con agua desionizada y almacénela de acuerdo a las instrucciones del fabricante.


\subsubsection{Sensor de conductividad eléctrica (CE) }
La salinidad de un suelo o agua, se refiere a la cantidad de sales presentes en solución, y puede ser estimada indirectamente mediante la medición de la conductividad eléctrica (CE). 
El valor de CE es influenciado por la concentración y composición de las sales disueltas. 
A mayor valor de CE, mayor es la salinidad presente. 
Es importante considerar que todos los fertilizantes inorgánicos son sales y por lo mismo tienen un efecto directo sobre la CE.

La salinidad es un fenómeno indeseable ya que afecta el crecimiento de las plantas  de varias maneras y por lo mismo, un aumento en la CE traerá como consecuencia una disminución de rendimiento.

\subsection{Oxígeno disuelto (OD)}
\subsubsection{¿Qué es el oxígeno disuelto?}
La concentración de oxígeno disuelto (OD) en agua es extremadamente importante en la naturaleza, al igual que para los ambientes artificiales.  
En océanos, lagos, ríos, y otros grandes cuerpos de agua, el oxígeno disuelto es esencial para el crecimiento y desarrollo de la vida acuática.  
Sin oxígeno, el agua puede volverse tóxica debido al decaimiento de la materia orgánica por bacterias anaeróbicas. 
En un ambiente industrial, el agua debe contener al menos 2 mg/L de oxígeno para proteger a las tuberías de corrosión.  
De cualquier manera, los sistemas de agua de calderas, en muchos casos no pueden contener más que 10 mg/L de oxígeno disuelto.

\subsubsection{¿Porque es importante el oxígeno disuelto?}
Los niveles de OD pueden ayudar a indicar la salud de un cuerpo de agua. 
Si los niveles de OD están dentro de la media o más altos, el agua es un buen ambiente para una amplia variedad de vida acuática. 
Si los niveles de OD son bajos, indica la presencia de contaminantes en el agua. 
Algunas especies de vida acuática pueden subsistir en agua con un amplio rango de OD; pero otras fallecen a bajos niveles de OD.

Oxígeno disuelto
Se espera que las lecturas de OD tengan una amplia fluctuación si la fuente de agua cuenta con abundante vida vegetal. 
Esto se debe al proceso de fotosíntesis. 
Ya que existe una menor actividad fotosintética en las noches, cuando la luz no está presente, y que tanto plantas como animales seguirán consumiendo oxígeno a través de la respiración, los niveles de OD en las horas de la mañana serán mucho menores a los de otras horas del día. 
Una vez inicia la fotosíntesis, los niveles de OD incrementarán. 
Este es un buen ejemplo de los beneficios de medir los parámetros varias veces durante el día. 
Si únicamente se realiza la medición de OD antes del amanecer, se llegará a una conclusión inadecuada sin importar la salud del cuerpo de agua.

Mientras que los niveles de OD se influencian particularmente por la actividad fotosintética, una gran cantidad de OD se obtiene de la mezcla del OD y el agua. 
Esto sucede si el agua es turbulenta en grandes cuerpos de agua. 
La turbulencia incrementa el área superficial del agua, para que el oxígeno atmosférico puede mezclarse más fácilmente. 
El aire tiene una concentración de oxígeno que es 20 veces mayor que la concentración de oxígeno en el agua. 
La diferencia de concentración resulta en el oxígeno atmosférico disolviéndose en agua cuando las dos se encuentran. 
Si hay más superficie de agua en su interfaz, entonces más oxígeno del aire se absorberá.

Otros factores que influencian los niveles de OD son la temperatura y los vertimientos. 
El oxígeno se disuelve más fácilmente en agua fría, y el agua fría tiene la capacidad de mantener mayores cantidades de gas que el agua tibia, así que el nivel de OD disuelto disminuye a medida que el agua se calienta. 
Los vertimientos pueden incluir desechos orgánicos o contaminantes creados por el hombre; en ambos casos, los organismos en el agua deben usar oxígeno en el proceso de descomponer estos contaminantes.  
También, los desechos orgánicos pueden llevar al crecimiento de vegetación acuática.  
Cuando las plantas mueren al final del lapso de crecimiento, grandes cantidades de oxígeno disuelto se consumen al descomponerse.

\subsubsection{¿Cómo podemos medir el oxígeno disuelto?}
Las concentración de oxígeno disuelto se reporta en miligramos de gas por litro de agua, mg/L. (La unidad equivalente a mg/L es equivalente a partes por millón=ppm)  

Las lecturas de OD se realizan habitualmente utilizando una sonda y un medidor.

 
Es importante realizar lecturas de OD en varios momentos del día, y a varias profundidades. 
Las mediciones le darán una visión general de los niveles de OD en el cuerpo de agua que investiga. 
Cómo sucede con todos los parámetros de calidad de agua, este parámetro debe monitorearse en el tiempo. 
Esto producirá suficiente información para identificar y evaluar tendencias.

\subsubsection{Calibración de OD en campo}
El contenido de oxígeno disuelto (OD) en el agua se mide usando un electrodo con membrana. 
Desafortunadamente, cepillos u otros objetos de limpieza pueden dañar la membrana, así que remplazar la tapa de la membrana y el electrolito es la mejor manera de realizar un mantenimiento periódico. 
Si bien es más fácil calibrar el sensor de OD antes de estar en campo, es mejor calibrar la sonda en campo pues las diferencias de altitud y presión barométrica entre la calibración y la lectura pueden resultar en errores. 
Asegúrese de verificar que las lecturas de presión barométrica, conductividad, y temperatura sean correctas.

\subsubsection{Procedimiento de calibración (100\%)}:

Llene un beaker de calibración con agua (como alternativa ubique una esponja húmeda o una toalla de papel húmeda en el fondo del contenedor usado para la calibración)
Ajuste ligeramente la sonda en el beaker de calibración para evitar escape de humedad. 
Asegúrese de no humedecer su sensor de OD por la evaporación en el sensor de temperatura pues esto puede influenciar las lecturas de la sonda durante la calibración.
Permita al contenedor saturarse con vapor de agua (por aproximadamente 10 a 15 minutos). 
Durante este periodo, encienda el instrumento para permitir que la sonda de O.D. caliente
Confirme el punto de calibración cuando la lectura sea estable, o cuando los dígitos no cambien por al menos 5 segundos.
La calibración esta completa. 
Enjuague la sonda con agua desionizada y almacénela de acuerdo a las instrucciones del fabricante.

\subsubsection{Procedimiento de calibración (0\%):}
Llene el beaker con suficiente solución 0\% OD para cubrir la unión del electrodo (cerca de 75 mL de un beaker de 100 mL)
Sumerja la sonda de OD en la solución.
Confirme el punto de calibración cuando la lectura sea estable, o cuando los dígitos no cambien por al menos 5 segundos.
La calibración esta completa. 
Enjuague la sonda con agua desionizada y almacénela de acuerdo a las instrucciones del fabricante. 
Asegúrese de enjuagar toda la solución $0\%$ OD para que esta no afecte las mediciones de sus muestras.

\subsubsection{Sensor de oxigeno disuelto (DO):}
Toda la vida acuática depende de la disponibilidad de oxígeno disuelto (DO) en el agua. 
Mientras que los organismos terrestres viven en una atmósfera compuesta aproximadamente de un 20\% de oxigeno, los organismos acuáticos sobreviven con una cantidad de oxigeno considerablemente menor. 
La solubilidad del oxigeno en agua dulce varia entre 14.6 mg/L a 0$^{\circ}$C hasta aproximadamente 7 mg/L a 35$^{\circ}$C bajo una presión de 760 mmHg. La concentración de oxígeno disuelto en agua está determinada por la ley de Henry, que describe la relación de equilibrio entre la presión parcial de oxígeno atmosférico y la concentración de oxígeno en agua. 
Otros factores que influyen la concentración de oxígeno disuelto en agua son: la presión atmosférica (y por lo tanto la altitud sobre el nivel del mar), el contenido de sales en el agua, y la temperatura del agua. 
El contenido de oxigeno disuelto en cuerpos de agua puede disminuir significativamente por efecto de la respiración, especialmente la microbiana, resultante de la degradación de compuestos orgánicos. 

Traducido eso a un sensor cual permita detectar estos factores, se tiene la sonda galvánica de oxigeno disuelto que consta de una membrana de politetrafluoroetileno (PTFE), un ánodo bañado en un electrolito y un cátodo. 
Las mol\'eculas de oxígeno se desactivan a través de la membrana de la sonda a una velocidad constante (sin la membrana, la reacción es rápida). 
Una vez el las mol\'eculas de oxigeno han atravesado la membrana se reducen en el cátodo y se produce una pequeña tensión. 
Si no hay moléculas de oxígeno presentes, la sonda emitiría 0 mV. 
A medida que aumenta el ox\'igeno, tambi\'en lo hace la salida de mV de la sonda. La sonda emite un voltaje diferente en presencia de ox\'igeno. Lo \'unico que es constante es que 0mV = 0 ox\'igeno []

emitite

%%%%%%%%%%%%%%%%%%%%%%%%%%%%%%%%%%%%%%%%%%%%%%%%%%%%%%%%%%%%%%%%%%%%%%%%%%%%%%%%%%%%%%%%%%%%%%%%%%%%%%%%%%%%%%%%%%%%%%%%%%


%PONER IMAGEN




\section{Muestreo de Aguas.}
%%[RR] información de Regina  



%% Revolución 222 tipos de aguas
%%
%%Coerencia de resiltado
%% Fiabilidad 
%% Estándar, regulador de transporte 
%%