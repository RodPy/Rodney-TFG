\chapter[Capítulo 5. Conclusi\'on y Trabajos Futuros]{ Conclusión y Trabajos Futuros}
\pagestyle{fancy}



Se logro dise\~nar un sonda con la capacidad de alojamiento de los sensores de conductividad el\'ectrica, potencial de hidr\'ogeno, oxigeno disuelto, potencial de oxigeno reducci\'on, temperatura, salinidad, y componentes elect\'onicos .
Empleando t\'ecnicas de mecanizado, se logro fabricar la sonda en material de nylon.
Se logro dotar de funcionalidad para medici\'on a multinivel mediante el uso motores el\'ectricos y con control programable del usuario.
Se logro implementar con la herramienta nodeRed una interfaz visual gr\'afica, que despliegue toda la informaci\'on de los sensores y f\'acil e intuitiva.
Se efectuaron varias pruebas de campo en el lago Ypakarai, laguna Pyta, laboratorio sistemas distribuidos y laboratorios de qu\'imica FIUNA. 