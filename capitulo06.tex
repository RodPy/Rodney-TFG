\chapter[Capítulo 6. ]{}
\pagestyle{fancy}

% %%%%%%%%%%%%%%%%%% CAPITULO AUXILIAR 

Borrador

El t/'ermino humedal incluye todos los cuerpos de agua l/'enticos poco profundos, temporales o permanentes, donde la luz llega al fondo permitiendo el desarrollo.\cite{quintana_new_2016} 


%% Calidad de Ahuas
En el 


\begin{table}[]
\begin{tabular}{|l|ccc|ccc}
\hline
Muestra 2                                                                       & \multicolumn{3}{c|}{LSD}                                                                                                                                 & \multicolumn{3}{c|}{QCA}                                                                                                                                                     \\ \hline
\textbf{\begin{tabular}[c]{@{}l@{}}Análisis de datos \\ obtenidos\end{tabular}} & \multicolumn{1}{c|}{\begin{tabular}[c]{@{}c@{}}T \\ °C\end{tabular}} & \multicolumn{1}{c|}{pH}      & \begin{tabular}[c]{@{}c@{}}CE\\ $\mu S/cm$\end{tabular} & \multicolumn{1}{c|}{\begin{tabular}[c]{@{}c@{}}T \\ °C\end{tabular}} & \multicolumn{1}{c|}{pH}     & \multicolumn{1}{c|}{\begin{tabular}[c]{@{}c@{}}CE\\ $\mu S/cm$\end{tabular}} \\ \hline
Desviación Estandar                                                             & \multicolumn{1}{c|}{0.291}                                           & \multicolumn{1}{c|}{0.165}   & 3.545                                              & \multicolumn{1}{c|}{0.354}                                           & \multicolumn{1}{c|}{0.167}  & \multicolumn{1}{c|}{1.125}                                              \\ \hline
Varianza                                                                        & \multicolumn{1}{c|}{0.0845}                                          & \multicolumn{1}{c|}{0.027}   & 12.57                                              & \multicolumn{1}{c|}{0.125}                                           & \multicolumn{1}{c|}{0.0279} & \multicolumn{1}{c|}{1.267}                                              \\ \hline
\begin{tabular}[c]{@{}l@{}}Coeficiente de \\ Variación\end{tabular}             & \multicolumn{1}{c|}{1.287}                                           & \multicolumn{1}{c|}{2.616}   & 9.816                                              & \multicolumn{1}{c|}{1.582}                                           & \multicolumn{1}{c|}{2.379}  & \multicolumn{1}{c|}{2.201}                                              \\ \hline
Error Estándar                                                                  & \multicolumn{1}{c|}{0.075}                                           & \multicolumn{1}{c|}{0.0426}  & 0.915                                              & \multicolumn{1}{c|}{0.0914}                                          & \multicolumn{1}{c|}{0.043}  & \multicolumn{1}{c|}{0.290}                                              \\ \hline
Covarianza                                                                      & \multicolumn{1}{c|}{0.0564}                                          & \multicolumn{1}{c|}{-0.351}  & 0.157                                              &                                                                      &                             &                                                                         \\ \cline{1-4}
Correlación                                                                     & \multicolumn{1}{c|}{0.5483}                                          & \multicolumn{1}{c|}{-0.0096} & 0.628                                              &                                                                      &                             &                                                                         \\ \cline{1-4}
\end{tabular}
\end{table}







\begin{table}[]
\begin{tabular}{clll}
\hline
\textbf{Sector}                                                         & \multicolumn{1}{c}{\textbf{Actividad}}                                                      & \multicolumn{1}{c}{\textbf{Efectos}}                                                                 & \multicolumn{1}{c}{\textbf{Área}}                                                                        \\ \hline
\multirow{6}{*}{\begin{tabular}[c]{@{}c@{}}Centro\\ Oeste\end{tabular}} & Urbana e industrial                                                                         & \begin{tabular}[c]{@{}l@{}}Contaminación superficial y \\ \\ acuíferos.\end{tabular}                 & \begin{tabular}[c]{@{}l@{}}REMA (Región Metropolitana \\ de Asunción)\\ Cuenca de Ypacarai.\end{tabular} \\
                                                                        & Mataderos y frigoríficos                                                                    & Aguas superficiales                                                                                  & REMA                                                                                                     \\
                                                                        & \multirow{2}{*}{\begin{tabular}[c]{@{}l@{}}Expansión Urbana \\ \\ desordenada\end{tabular}} & Destrucción de fuentes de agua                                                                       & \begin{tabular}[c]{@{}l@{}}Cuenca de Ypacarai\\ Cuenca del Ypoa\end{tabular}                             \\
                                                                        &                                                                                             & Aguas superficiales                                                                                  & REMA                                                                                                     \\
                                                                        & \begin{tabular}[c]{@{}l@{}}Cementera\\ Metalurgica\end{tabular}                             & \begin{tabular}[c]{@{}l@{}}Contaminación superficial y \\ \\ acuíferos\end{tabular}                  & \begin{tabular}[c]{@{}l@{}}REMA\\ Villa Hayes\end{tabular}                                               \\
                                                                        & Prácticas agrícolas .                                                                       & \begin{tabular}[c]{@{}l@{}}Colmatación por erosión y \\ \\ contaminación por pesticidas\end{tabular} & Todos los Deptos                                                                                         \\ \cline{2-4} 
\end{tabular}
\hfill \textbf{Fuente:} Extra\'ido de \cite{salas-duenas-2015}.
\end{table}





%%%%%%%%%%%%%%%%%%%%%%%%%%%%%%CAP 3


\subsection{}section[Descripción General del Sistema]{Descripción General del Sistema}

El modelo del sistema, está compuesto por varios subsistemas o bloques escalables, independientes y que se pueden comunican entre s\'i, mediante un un intermediario que se encarga de coordinar los datos entre los distintos bloques. El sistema general se podr\'ia dividir en tres segmentos principales, por un lado el subsistema central \textit{la sonda}, de mayor complejidad que se encarga de recepci\'on y transmisi\'on de datos y nexo con los otros subsistemas, ejecuta el muestreo con los sensores, almacena en una base local los datos, la \textit{estacion remota}, la sala de monitoreo donde se puede visualizar los datos recibido de la sonda y realizar configuraciones de muestreo, los subsistemas menos complejos se denominar perif\'ericos, son bloques auxiliares escalables en medida que sean necesarios, y de esa forma aumentar las funcionalidades del sistema, en el presente TFG de desarrollaron e implementaron los perif\'ericos que ayuden para el muestreo a multiniveles como ser perif\'ericos de \textit{batimetr\'ia}, para conocer la profundidad del lugar de muestreo y perif\'erico de \textit{gr\'ua} que le otorga a la sonda la posibilidad de desplazamiento vertical para pueda tomar muestras a multiniveles y el perif\'erico \textit{ASV}, provee la ubicaci\'on georeferenciada del punto de muestreo y otorga a la sonda la posibilidad de desplazamiento horizontal sobre toda la superficie a ser analizada. Los modos de operaci\'on pueden ser autom\'aticos, donde ejecuta un algoritmo en sincron\'ia con el ASV o manual donde el usuario desde la estaci\'on remota puede configurar los par\'ametros que requiera.  
El diagrama de funcionamiento del sistema se pude apreciar en la Figura \ref{fig:3.1}.

%%%%https://es.overleaf.com/project/5ca9254d2df4572fd3ca381a
asd 


\subsubsection{Sonda}








\section{Aspectos Generales}
El desarrollo de la sonda implic\'o una serie de desaf\'ios t\'ecnicos, por ese motivo esa secci\'on se organiza de la siguiente manera

\begin{itemize}
    \item Dise\~no Mec\'anico: abarca todos los criterios considerados al dise\~no de la sonda y sistema de despliegue. 
    \item Fabricaci\'on Mec\'anica: abarca las t\'ecnicas y materiales empleadas para la fabricaci\'on de los componentes.
    \item Programaci\'on e Integraci\'on de los sensores: en esta secci\'on abarca lo referente a los distintos script desarrollados para lograr el funcionamiento del dispositivo.
\end{itemize}
\section{Dise\~no Mec\'anico}
Para el proceso de dise\~no se siguieron los procedimientos de XXX donde agrupan el proceso de dise\~no en las siguientes etapas:
Etapas del dise\~no: 

\subsection{Sonda}
\subsubsection{Conceptualizaci\'on}
Conceptualmente se definen los requerimientos m\'inimos para el correcto funcionamiento de la sonda.
\begin{itemize}
    \item versatilidad: su implementación debe de ser independiente a la dispositivo de soporte, pudiendo operar en cualquier medio de transporte.
    \item 
    \item Energ\'ia: la donde deber\'a tener autonom\'ia energ\'etica para sus operaciones de monitoreo,  
\end{itemize}
la donde deberá de contener en su interior al menos cinco sensores  (ph,CE,OD,OPR,T), adem\'as de toda la electr\'onica necesaria para la operaci\'on de estos. La sonda deber\'a tener autonom\'ia energ\'etica y poder sumergirse hasta una profundidad m\'inima de 5 metros , teniendo en cuenta que la profundidad promedio del lago Ypakarai es de 3 metros [\cite{hidrologiaItaipu}].
   
\subsubsection{An\'alisis}
En base a las conceptualizaci\'on de la secci\'on aterior y los materiales disponibles en el pa\'is se concluyen  :


\subsection{Despliegue}
\subsubsection{Conceptualizaci\'on}

deber\'a brindar soporte para que la sonda pueda hacer mediciones a varios niveles, dise\~no simple y adaptativo para ser instalado en embarcaciones de investigaci\'on.

\subsubsection{An\'alisis}
En base a las conceptualizaci\'on de la secci\'on aterior y los materiales dispobibles en el pa\'is se concluyen  :
\begin{itemize}
    \item Sonda: .
    \item Despliegue: deber\'a brindar soporte para que la sonda pueda hacer mediciones a varios niveles, dise\~no simple y adaptativo para ser instalado en embarcaciones de investigaci\'on.
\end{itemize}

\subsection{Diseño, Mecanizado de Piezas}



\section{Fabricaci\'on}





%-----------------------------------------------------------------------------




         

\subsection{Referencias Comerciales.}
Las pocas empresas que se dedican al rubro de la construcción de invernaderos Hidropónicos, las que se dedican al riego automatizado o las de ferti-riego ofrecen soluciones a proyectos de gran envergadura, debiéndose especificar siempre las hectáreas de plantación que se desean controlar para que tomen en cuenta el proyecto, lo diseñen y así puedan enviar un presupuesto al cliente interesado.
Según revisión de existencia local no existe un producto comercial del tipo de este trabajo dentro del territorio paraguayo.\textbf{}





%%%%%%%%%%%%%%%%%%%%%%%%%%%%%%%%%%%%%%%%%%%%%%%%%%%%%%%%%%%%%%%%%%%%%%%%%%%%%%%%%%%%%%%%%%%%%%%%
%%%%%%%%%%%%%%%%%%%% ANOTADOR CAP 1 %%%%%%%%%%%%%%%%%%%%%%%%%%%%%%%%%%

Monitoreo del agua potable: los parámetros químicos comunes incluyen pH, nitratos y oxígeno disuelto. 

La medición de O2 (o DO) es un indicador importante de la calidad del agua. 
Los cambios en los niveles de oxígeno disuelto indican la presencia de microorganismos de aguas residuales, escorrentías urbanas o agrícolas o descargas de fábricas. 

Un nivel adecuado de ORP minimiza la presencia de microorganismos como  E. coli, Salmonella, Listeria . 

Los niveles de turbidez por debajo de 1 NTU indican la pureza adecuada del agua potable.

Detección de fugas químicas en ríos : pH extremo o valores bajos de OD indican derrames químicos debido a problemas en la planta de tratamiento de aguas residuales o en la tubería de suministro.

Medición remota de piscinas : la medición del potencial de oxidación-reducción (ORP), el pH y los niveles de cloro del agua puede determinar si la calidad del agua en piscinas y spas es suficiente para fines recreativos.

Niveles de contaminación en el mar : la medición de los niveles de temperatura, salinidad, pH, oxígeno y nitratos proporciona información para los sistemas de detección de calidad en el agua de mar.

Prevención de la corrosión y depósitos de cal:  Controlando la dureza del agua podemos evitar la corrosión y depósitos de cal en lavavajillas y dispositivos de tratamiento de agua como calentadores. 

La dureza del agua depende de: pH, temperatura, conductividad y concentraciones de calcio (Ca + ) / magnesio (Mg 2+ ).

Cultivo de abetos / Monitoreo de tanques de peces / Criadero / Acuicultura / Acuaponía: Medición de las condiciones del agua de animales acuáticos como caracoles, peces, cangrejos de río, camarones o langostinos en tanques. 

Los valores importantes son el pH, el oxígeno disuelto (DO), el amoníaco (NH 4 ), el nitrato (NO 3 - ), el nitrito (NO 2 - ) y la temperatura del agua.

Hidroponía : las plantas que toman los nutrientes directamente del agua necesitan un pH preciso y niveles de oxígeno en el agua (OD) para obtener el máximo crecimiento.

Las sondas del sensor miden más de 12 parámetros químicos y físicos de la calidad del agua, como pH, nitratos (NO3), iones disueltos (fluoruro (F - ), calcio (Ca 2+ ), nitrato (NO 3 - ), cloruro (Cl - ), Yoduro (I - ), Cúprico (Cu 2+ ), Bromuro (Br - ), Plata (Ag + ), Fluoroborato (BF 4 - ), Amoníaco (NH 4 ), Litio (Li + ), Magnesio (Mg 2+ ) , Nitrito (NO 2- ), Perclorato (ClO 4 ), Potasio (K + ), Sodio (Na +) oxígeno disuelto (OD), conductividad (salinidad), potencial de oxidación-reducción (ORP), turbidez, temperatura, etc.

Los contaminantes pueden detectarse y tratarse en tiempo real para garantizar una buena calidad del agua en toda la red de suministro de agua. 
Los valores extremos de pH pueden indicar derrames de productos químicos, problemas en la planta de tratamiento o problemas en las tuberías de suministro. 
Los niveles bajos de OD pueden indicar la presencia de microorganismos debido a escorrentías urbanas / agrícolas o derrames de aguas residuales. 
El ORP mide qué tan bien está funcionando la desinfección del agua.






%%%%%%%%%%%%%%%%%%%%%%%%%%%%%%%%%%%%%%%%%%%%%%%%%%%%%%%%%%%%%%%%%%%%%%%%%%%%%%%%%%%%%%%%%%%%%%%%%%%%%%%%%%%%%%%%%%%%%%%%%%


%Desde acá considerar pasar todo al cap 3
\section{T\'ecnicas de fabricaci\'on mecanicas}
% http://www.scielo.org.co/scielo.php?script=sci_arttext&pid=S0120-56092006000300014
% https://books.google.com.co/books?id=tcV0l37tUr0C&printsec=frontcover#v=onepage&q&f=false
% http://ve.scielo.org/scielo.php?script=sci_arttext&pid=S1316-48212014000400003
% https://www.3dnatives.com/es/los-mejores-libros-impresion-3d-22042016/#!
% https://es.wikipedia.org/wiki/Impresi%C3%B3n_3D
% https://esvigrocircuitos.webnode.mx/_files/200000118-58b4259ae5/Identificaci%C3%B3n%20general%20de%20sistemas%20y%20t%C3%A9cnicas%20de%20fabricaci%C3%B3n.pdf
% % https://www.redalyc.org/jatsRepo/944/94454631006/html/index.html
% https://books.google.com.py/books?id=gBuyDwAAQBAJ&printsec=frontcover&dq=impresion+3D&hl=es-419&sa=X&redir_esc=y#v=onepage&q&f=false
% https://books.google.com.py/books?id=JQ9djwEACAAJ&dq=impresion+3D&hl=es-419&sa=X&redir_esc=y
% https://books.google.com.py/books?id=fvJ9DwAAQBAJ&pg=PT8&dq=impresion+3D&hl=es-419&sa=X&ved=2ahUKEwi5ptXwte33AhXhmZUCHZgHD3wQ6AF6BAgHEAI#v=onepage&q=impresion%203D&f=false

\section{Dise~/no de interfaz}
% https://nodered.org/
% https://docs.python.org/3/library/tkinter.html

\section{Base de Datos}
% % https://www.oracle.com/mx/database/what-is-database/
% https://www.postgresql.org/
% https://www.sqlite.org/index.html
% https://books.google.com.py/books?id=HmnHeZ1wsvwC&printsec=frontcover&dq=bases+de+datos&hl=es-419&sa=X&redir_esc=y#v=onepage&q=bases%20de%20datos&f=false
% https://books.google.com.py/books?id=EwcuBwAAQBAJ&printsec=frontcover&dq=bases+de+datos&hl=es-419&sa=X&redir_esc=y#v=onepage&q=bases%20de%20datos&f=false
% https://books.google.com.py/books?id=LDOGzQEACAAJ&dq=bases+de+datos+sqlite&hl=es-419&sa=X&redir_esc=y

\section{Comunicaci\'on}
% https://mecatronicaexperimental.weebly.com/protocolos-de-comunicacioacuten.html
% https://sites.google.com/site/protocolosse/
% https://cs.uns.edu.ar/materias/se/2019/descargas/teoria/clase08-comunicacion-slides.pdf
% https://www.virtual-serial-port.org/es/articles/serial-communication-in-embedded-development/
% https://1library.co/article/protocolos-de-comunicaci%C3%B3n-en-sistemas-embebidos.zlnpvxrq
% http://sedici.unlp.edu.ar/bitstream/handle/10915/72131/Tesis.%20Protocolos%20de%20comunicaci%C3%B3n%20entre%20microcontroladores.pdf-PDFA.pdf?sequence=2&isAllowed=y
% https://books.google.com.py/books?id=5gSQhn0LCOsC&pg=PA117&dq=1wire&hl=es-419&sa=X&ved=2ahUKEwiT8ey9te33AhVeI7kGHcCbCXUQ6AF6BAgKEAI#v=onepage&q=1wire&f=false
% https://books.google.com.py/books?id=aGVTAAAAMAAJ&q=i2c&dq=i2c&hl=es-419&sa=X&ved=2ahUKEwjissm4te33AhV_BLkGHTVsC3kQ6AF6BAgJEAI
